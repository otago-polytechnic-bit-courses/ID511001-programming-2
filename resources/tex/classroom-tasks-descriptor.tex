% Author: Grayson Orr
% Course: ID511001: Programming 2

\documentclass{article}
\author{}

\usepackage{graphicx}
\usepackage{wrapfig}
\usepackage{enumerate}
\usepackage{hyperref}
\usepackage[margin = 2.25cm]{geometry}
\usepackage[table]{xcolor}
\usepackage{soul}
\usepackage{fancyhdr}
\usepackage{fancyvrb}
\hypersetup{
  colorlinks = true,
  urlcolor = blue
}
\setlength\parindent{0pt}
\pagestyle{fancy}
\fancyhf{}
\rhead{College of Engineering, Construction \& Living Sciences\\Bachelor of Information Technology}
\lfoot{Classroom Tasks: Unity Game Engine Research\\Version 1, Semester One, 2023}
\rfoot{\thepage}
 
\begin{document}

\begin{figure}
    \centering
    \includegraphics[width=50mm]{../../resources/img/logo.png}
\end{figure}

\title{College of Engineering, Construction \& Living Sciences\\Bachelor of Information Technology\\ID511001: Programming 2\\Level 5, Credits 15\\\textbf{Classroom Tasks: Unity Game Engine Research}}
\date{}
\maketitle

\section*{Assessment Overview}
In this assessment, you will develop two projects using the \textbf{Unity} game engine as well answer questions pertaining to different concepts in the \textbf{Unity} game engine. 

\section*{Learning Outcomes}
At the successful completion of this course, learners will be able to:
\begin{enumerate}
    \item Build interactive, event-driven GUI applications using pre-built components.
    \item Declare \& implement user-defined classes using encapsulation, inheritance \& polymorphism.
\end{enumerate}

\section*{Assessments}
\renewcommand{\arraystretch}{1.5}
\begin{tabular}{|c|c|c|c|}
	\hline
	\textbf{Assessment}                                 & \textbf{Weighting} & \textbf{Due Date}            & \textbf{Learning Outcomes} \\ \hline
	\small Project 1 (C\# Console App): Learner Gradebook  & \small 25\%        & \small 26-04-2023 (Wednesday at 4.59 PM)   & \small 1 \& 2                   \\ \hline
	\small Project 2 (C\# Windows Forms App): Pong & \small 35\%        & \small 14-06-2023 (Wednesday at 4.59 PM)   & \small 1 \& 2                   \\ \hline
	\small Theory Examination                        & \small 30\%        & \small 21-06-2023 (Wednesday at 4.45 PM)  & \small 1 \& 2                   \\ \hline
	\small Classroom Tasks                       & \small 10\%        & \small 07-06-2023 (Wednesday at 4.59 PM)  & \small 1 \& 2                   \\ \hline
\end{tabular} 

\section*{Conditions of Assessment}
You will complete this assessment during your learner-managed time. However, there will be time during class to discuss the requirements \& your progress on this assessment. This assessment will need to be completed by \textbf{Wednesday, 07 June 2023} at \textbf{4.59 PM}.

\section*{Pass Criteria}
This assessment is criterion-referenced (CRA) with a cumulative pass mark of \textbf{50\%} over all assessments in \textbf{ID511001: Programming 2}.

\section*{Authenticity}
All parts of your submitted assessment \textbf{must} be completely your work. If you use code snippets from \textbf{GitHub}, \textbf{StackOverflow} or other online resources, you \textbf{must} reference it appropriately using \textbf{APA 7th edition}. Provide your references in the \textbf{README.md} file in your repository. Failure to do this will result in a mark of \textbf{zero} for this assessment. 

\section*{Policy on Submissions, Extensions, Resubmissions \& Resits}
The school's process concerning submissions, extensions, resubmissions \& resits complies with \textbf{Te Pūkenga} policies. Learners can view policies on the \textbf{Te Pūkenga} website located at \href{https://www.op.ac.nz/about-us/governance-and-management/policies}{https://www.op.ac.nz/about-us/governance-and-management/policies}.

\section*{Submission}
You \textbf{must} submit all app files via \textbf{GitHub Classroom}. Here is the URL to the repository you will use for your submission – \href{https://classroom.github.com/a/YOLzkboo}{https://classroom.github.com/a/YOLzkboo}.  Create a \textbf{.gitignore} \& add the ignored files in this resource - \href{https://raw.githubusercontent.com/github/gitignore/main/Unity.gitignore}{https://raw.githubusercontent.com/github/gitignore/main/Unity.gitignore}. The latest app files in the \textbf{classroom-tasks-unity} branch will be used to mark against the \textbf{Functionality} criterion. Please test before you submit. Partial marks \textbf{will not} be given for incomplete functionality. Late submissions will incur a \textbf{10\% penalty per day}, rolling over at \textbf{5:00 PM}.

\section*{Extensions}
Familiarise yourself with the assessment due date. Contact the course lecturer before the due date if you need an extension. If you require more than a week's extension, you will need to provide a medical certificate or support letter from your manager.

\section*{Resubmissions}
Learners may be requested to resubmit an assessment following a rework of part/s of the original assessment. Resubmissions are to be completed within a negotiable short time frame \& usually \textbf{must} be completed within the timing of the course to which the assessment relates. Resubmissions will be available to learners who have made a genuine attempt at the first assessment opportunity \& achieved a \textbf{D grade (40-49\%)}. The maximum grade awarded for resubmission will be \textbf{C-}.

\section*{Resits}
Resits \& reassessments \textbf{are not} applicable in \textbf{ID511001: Programming 2}.

\section*{Instructions}
You will need to submit an app \& documentation that meet the following requirements:\\

\subsection*{Functionality - Learning Outcomes 1 \& 2 (50\%)}
Complete the two projects as outlined in the \textbf{classroom-tasks-unity.md} file.

\subsection*{Questions - Learning Outcomes 1 \& 2 (50\%)}
\begin{itemize} 
    \item Provide answers to the following in your repository \textbf{README.md} file:
    \begin{itemize}
        \item Describe some best practices for organising your Unity project.
        \item Explain the concept of \textbf{GameObjects} \& \textbf{Components}, \& how they are related.
        \item Explain the concept of \textbf{Prefabs}. What are they, \& how do they contribute to the efficiency \& reusability?
        \item Explain the concept of game object hierarchy. How can you organise \& structure game objects using parent-child relationships? Provide an example of a situation where using hierarchy can be beneficial.
        \item In the game \textbf{Pong}, how would you implement the collision detection between the ball \& the paddles using colliders?
    \end{itemize}
\end{itemize}

\end{document}