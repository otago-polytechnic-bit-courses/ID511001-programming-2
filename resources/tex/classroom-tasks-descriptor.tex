% Author: Grayson Orr
% Course: ID511001: Programming 2

\documentclass{article}
\author{}
 
\usepackage{fontspec}
\setmainfont{Arial}

\usepackage{graphicx}
\usepackage{wrapfig}
\usepackage{enumerate}
\usepackage{hyperref}
\usepackage[margin = 2.25cm]{geometry}
\usepackage[table]{xcolor}
\usepackage{soul}
\usepackage{fancyhdr}
\usepackage{fancyvrb}
\hypersetup{
  colorlinks = true,
  urlcolor = blue
}
\setlength\parindent{0pt}
\pagestyle{fancy}
\fancyhf{}
\rhead{College of Engineering, Construction and Living Sciences\\Bachelor of Information Technology}
\lfoot{Classroom Tasks \\Version 3, Semester Two, 2024}
\rfoot{\thepage}
 
\begin{document}

\begin{figure}
    \centering
    \includegraphics[width=50mm]{../../resources/img/logo.png}
\end{figure}

\title{College of Engineering, Construction and Living Sciences\\Bachelor of Information Technology\\ID511001: Programming 2\\Level 5, Credits 15\\\textbf{Classroom Tasks}}
\date{}
\maketitle

\section*{Assessment Overview}
In this assessment, you will \textbf{unit test} your \textbf{Project 2 Windows Form Application} using \textbf{C\#}. 

\section*{Learning Outcomes}
At the successful completion of this course, learners will be able to:
\begin{enumerate}
    \item Build interactive, event-driven GUI applications using pre-built components.
    \item Declare and implement user-defined classes using encapsulation, inheritance and polymorphism.
\end{enumerate}

\section*{Assessments}
\renewcommand{\arraystretch}{1.5}
\begin{tabular}{|c|c|c|c|}
	\hline
	\textbf{Assessment}                                 & \textbf{Weighting} & \textbf{Due Date}            & \textbf{Learning Outcomes} \\ \hline
	\small Project 1 & \small 25\%        & \small \small 13-11-2024 (Wednesday at 4.59 PM)  & \small 1, 2                   \\ \hline
	\small Project 2  & \small 35\%        & \small 27-09-2024 (Friday at 4.59 PM)   & \small 1, 2                   \\ \hline
	\small Theory Examination                        & \small 30\%        & \small 18-11-2024 (Monday at 12.00 PM)  & \small 1, 2                   \\ \hline
	\small Classroom Tasks                      & \small 10\%        & \small 27-09-2024 (Friday at 4.59 PM)  & \small 1, 2                   \\ \hline
\end{tabular} 

\section*{Conditions of Assessment}
You will complete this assessment during your learner-managed time. However, there will be time during class to discuss the requirements and your progress on this assessment. This assessment will need to be completed by \textbf{Wednesday, 18 September 2024} at \textbf{4.59 PM}.

\section*{Pass Criteria}
This assessment is criterion-referenced (CRA) with a cumulative pass mark of \textbf{50\%} over all assessments in \textbf{ID511001: Programming 2}.

\section*{Authenticity}
All parts of your submitted assessment \textbf{must} be completely your work. Do your best to complete this assessment without using an \textbf{AI generative tool}. You need to demonstrate to the course lecturer that you can meet the learning outcome(s) for this assessment. \\
 
 However, if you get stuck, you can use an \textbf{AI generative tool} to help you get unstuck, permitting you to acknowledge that you have used it. In the assessment's repository \textbf{README.md} file, please include what prompt(s) you provided to the \textbf{AI generative tool} and how you used the response(s) to help you with your work. It also applies to code snippets retrieved from \textbf{StackOverflow} and \textbf{GitHub}. \\
 
 Failure to do this may result in a mark of \textbf{zero} for this assessment. 

\section*{Policy on Submissions, Extensions, Resubmissions and Resits}
The school's process concerning submissions, extensions, resubmissions and resits complies with \textbf{Otago Polytechnic | Te Pūkenga} policies. Learners can view policies on the \textbf{Otago Polytechnic | Te Pūkenga} website located at \href{https://www.op.ac.nz/about-us/governance-and-management/policies}{https://www.op.ac.nz/about-us/governance-and-management/policies}.

\section*{Submission}
You \textbf{must} submit all application files via \textbf{GitHub Classroom}. Here is the URL to the repository you will use for your submission – \href{https://classroom.github.com/a/o7D2CGoa}{https://classroom.github.com/a/o7D2CGoa}. If you do not have not one, create a \textbf{.gitignore} and add the ignored files in this resource - \href{https://raw.githubusercontent.com/github/gitignore/main/VisualStudio.gitignore}{https://raw.githubusercontent.com/github/gitignore/main/VisualStudio.gitignore}. Create a branch called \textbf{classroom-tasks}. The latest application files in the \textbf{classroom-tasks} branch will be used to mark against the \textbf{Functionality} criterion. Please test before you submit. Partial marks \textbf{will not} be given for incomplete functionality. Late submissions will incur a \textbf{10\% penalty per day}, rolling over at \textbf{5:00 PM}.

\section*{Extensions}
Familiarise yourself with the assessment due date. Extensions will \textbf{only} be granted if you are unable to complete the assessment by the due date because of \textbf{unforeseen circumstances outside your control}. The length of the extension granted will depend on the circumstances and \textbf{must} be negotiated with the course lecturer before the assessment due date. A medical certificate or support letter may be needed. Extensions will not be granted on the due date and for poor time management or pressure of other assessments.

\section*{Resubmissions}
Learners may be requested to resubmit an assessment following a rework of part/s of the original assessment. Resubmissions are to be completed within a negotiable short time frame and usually \textbf{must} be completed within the timing of the course to which the assessment relates. Resubmissions will be available to learners who have made a genuine attempt at the first assessment opportunity and achieved a \textbf{D grade (40-49\%)}. The maximum grade awarded for resubmission will be \textbf{C-}.

\section*{Resits}
Resits and reassessments \textbf{are not} applicable in \textbf{ID511001: Programming 2}.

\section*{Instructions}
You will need to submit an application and documentation that meet the following requirements:\\

\subsection*{Functionality - Learning Outcome 1 (50\%)}
\begin{itemize}
    \item The application needs to open without code or file structure modification in \textbf{Visual Studio}.
    \item You need to create 20 \textbf{unit tests} covering the following:
    \begin{itemize}
        \item All \textbf{properties} in the \textbf{Institution} class. \textbf{Three} tests are expected.
        \item All \textbf{properties} in the \textbf{Person} class. \textbf{Three} tests are expected.
        \item All \textbf{methods} in the \textbf{CourseAssessmentMark} class. \textbf{Seven} tests are expected.
        \item The number of \textbf{Institution}, \textbf{Department} and \textbf{Course} objects after seeding. \textbf{Three} tests are expected.
        \item The salary of a \textbf{Lecturer}, \textbf{Senior Lecturer}, \textbf{Principal Lecturer} and \textbf{Associate Professor}. \textbf{Four} tests are expected.
    \end{itemize}
\end{itemize}

\subsection*{Code Quality and Best Practices - Learning Outcome 2 (45\%)}
\begin{itemize} 
    \item Appropriate naming of files, variables, methods and classes.
    \item Idiomatic use of values, control flow, data structures and in-built functions.
    \item Efficient algorithmic approach.
    \item Sufficient modularity.
    \item Each file has an \textbf{XML documentation comment} located at the top of the file. In the \textbf{assessment} directory of the \textbf{course materials} repository, you will find an \textbf{XML documentation comment} example in the \textbf{xml-documentation-comment.txt} file.
    \item Formatted code.
    \item No dead or unused code. 
\end{itemize}

\subsection*{Documentation and Git Usage - Learning Outcome 2 (5\%)}
\begin{itemize}
    \item Provide the following in your repository \textbf{README.md} file:
    \begin{itemize}
        \item How to run your unit tests?
    \end{itemize}
\end{itemize}

\subsection*{Additional Information}
\begin{itemize}
    \item \textbf{Do not} rewrite your \textbf{Git} history. It is important that the course lecturer can see how you worked on your assessment over time.
\end{itemize}

\end{document}