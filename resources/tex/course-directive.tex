% Author: Grayson Orr
% Course: ID511001: Programming 2

\documentclass{article}
\author{}

\usepackage{graphicx}
\usepackage{wrapfig}
\usepackage{enumerate}
\usepackage{hyperref}
\usepackage[margin = 1.5cm]{geometry}
\usepackage[table]{xcolor}
\hypersetup{
  colorlinks = true,
  urlcolor = blue
}
\setlength\parindent{0pt}

\begin{document}

\begin{figure}
	\includegraphics[width=50mm]{../img/logo.png}
\end{figure}

\title{Course Directive\\ID511001: Programming 2\\Semester Two, 2022}
\date{}
\maketitle

\section*{Course Information}
\begin{tabular}{ll}
	Level:        & 5 \\
	Credits:      & 15                                                             \\
	Prerequisite: & ID510001: Programming 1                                  \\
	Timetable:    & Stream A: Wednesday 1 PM D202 \& Friday 8 AM D202  \\
	& Stream B: Tuesday 3 PM D202 \& Thursday 8 AM D202  \\
	Tutorials:    & Monday \& Thursday 8.30 PM - 10.00 PM                                    
\end{tabular}

\section*{Teaching Staff}
\begin{tabular}{ll}
	Name:            & Grayson Orr                           \\
	Position:        & Lecturer \\
	Office Location: & D318                                 \\
	Email Address    & grayson.orr@op.ac.nz                    \\
\end{tabular}

\section*{Course Dates}
\begin{tabular}{ll}
	Term 1:             &  Monday 18 July - Friday 30 September \\
	Mid Semester Break: &  Monday 3 October - Friday 14 October     \\
	Term 2:             &  Monday 17 October - Friday 18 November      \\
\end{tabular}

\section*{Public Holidays \& Anniversary Days}
A list of public holidays \& anniversary days can be found here - \href{https://www.op.ac.nz/students/importantdates}{https://www.op.ac.nz/students/importantdates}

\section*{Aims}
To enable learners to build simple object-oriented (OO) applications and to identify situations that are most appropriate for OO implementation.

\section*{Learning Outcome}
At the successful completion of this course, learners will be able to:
\begin{enumerate}
	\item Build interactive, event-driven GUI applications using pre-built components.
	\item Declare \& implement user-defined classes using encapsulation, inheritance \& polymorphism.
\end{enumerate}

\section*{Assessments}
\renewcommand{\arraystretch}{1.5}
\begin{tabular}{|c|c|c|c|}
	\hline
	\textbf{Assessment}                                 & \textbf{Weighting} & \textbf{Due Date}            & \textbf{Learning Outcomes} \\ \hline
	\small Project 1: Pong & \small 25\%        & \small 17-10-2022 (Monday at 4.59 PM)   & \small 1 \& 2                   \\ \hline
	\small Project 2: Space Invaders & \small 35\%        & \small 04-11-2022 (Friday at 4.59 PM)   & \small 1 \& 2                   \\ \hline
	\small Theory Examination                        & \small 30\%        & \small 18-11-2022 (Friday at 4.59 PM)  & \small 1 \& 2                   \\ \hline
	\small Classroom Tasks                       & \small 10\%        & \small 21-10-2022 (Friday at 4.59 PM)  & \small 1 \& 2                   \\ \hline
\end{tabular}

\section*{Provisional Schedule}
\renewcommand{\arraystretch}{1.5}
\begin{tabular}{|c|c|c|c|}
	\hline
	\textbf{Week}                  & \textbf{Date Starting}            & \multicolumn{2}{c|}{\textbf{Topics}}                                                                                             \\ \hline
	\footnotesize 1/Tahi           & \footnotesize 18-07-2022 & \multicolumn{2}{c|}{\footnotesize Visual Studio IDE \& Interface Controls}    \\ \hline
	\footnotesize 2/Rua            & \footnotesize 25-07-2022 & \multicolumn{2}{c|}{\footnotesize Classes \& Objects}                   \\ \hline
	\footnotesize 3/Toru           & \footnotesize 01-08-2022 & \multicolumn{2}{c|}{\footnotesize Multiple Classes} \\ \hline
	\footnotesize 4/Whā            & \footnotesize 08-08-2022 & \multicolumn{2}{c|}{\footnotesize Encapsulation}                               \\ \hline
	\footnotesize 5/Rima           & \footnotesize 15-08-2022 & \multicolumn{2}{c|}{\footnotesize Graphics Class}                                                \\ \hline
	\footnotesize 6/Ono            & \footnotesize 22-08-2022 &  \multicolumn{2}{c|}{\footnotesize Enumerations \& Debugging}                            \\ \hline
	\footnotesize 7/Whitu          & \footnotesize 29-08-2022 & \multicolumn{2}{c|}{\footnotesize Inheritance \& Polymorphism}                                                   \\ \hline
	\footnotesize 8/Waru           & \footnotesize 05-09-2022 & \multicolumn{2}{c|}{\footnotesize Unity Game}                                                   \\ \hline
	\footnotesize 9/Iwa            & \footnotesize 12-09-2022 & \multicolumn{2}{c|}{\footnotesize  Unity Game }                                                                 \\ \hline
	\footnotesize 10/Tekau         & \footnotesize 19-09-2022 & \multicolumn{2}{c|}{\footnotesize Project 1 Work}                                                                 \\ \hline
	\footnotesize 11/Tekau mā tahi & \footnotesize 26-09-2022 & \multicolumn{2}{c|}{\footnotesize Project 1 Work}                                                                 \\ \hline
	\rowcolor{yellow} \multicolumn{4}{|c|}{\footnotesize Mid Term Break}                                                                                                                         \\ \hline
	\footnotesize 12/Tekau mā rua  & \footnotesize 17-10-2022 & \multicolumn{2}{c|}{\footnotesize Project 1 Code Defence }                                                                 \\ \hline
	\footnotesize 13/Tekau mā toru & \footnotesize 24-10-2022 & \multicolumn{2}{c|}{\footnotesize Project 2 Work}                                                     \\ \hline
	\footnotesize 14/Tekau mā whā  & \footnotesize 31-10-2022 & \multicolumn{2}{c|}{\footnotesize Project 2 Work} \\ \hline 
	\footnotesize 15/Tekau mā rima & \footnotesize 07-11-2022 & \multicolumn{2}{c|}{\footnotesize Project 2 Code Defence}                                                       \\ \hline
	\footnotesize 16/Tekau mā ono  & \footnotesize 14-11-2022 & \multicolumn{2}{c|}{\footnotesize Theory Examination}                                                         \\ \hline
\end{tabular}

\section*{Resources}

\subsection*{Software}
This paper will be taught using \textbf{Microsoft Visual Studio}. An installer for \textbf{Microsoft Visual Studio} is available - \href{https://visualstudio.microsoft.com/downloads}{https://visualstudio.microsoft.com/downloads}. Please refer any problems with downloads or installers to \textbf{Rob Broadley} in D205a.

\subsection*{Readings}
No textbook is required for this course. URLs to useful resources will be provided in the lecture notes.

\section*{Course Requirements \& Expectations}

\subsection*{Learning Hours}
This course requires \textbf{150 hours} of learning. This time includes \textbf{64 hours} of timetabled class time, \& \textbf{86 hours} of self-directed reading, preparation \& completion of assessments.

\subsection*{Criteria for Passing}
To pass this paper, you must achieve a cumulative pass mark of \textbf{50\%} over all assessments. There are no reassessments or resits. 

\subsection*{Attendance}
\begin{itemize}
	\item Learners are expected to attend all classes, including lectures \& labs.
	\item If you cannot attend for a few days for any reason, contact the course.
\end{itemize} 

\subsection*{Communication}
\textbf{Microsoft Outlook/Teams} are the official communication channels for this course. It is your responsibility to regularly check \textbf{Microsoft Outlook/Teams} \& \href{https://github.com/otago-polytechnic-bit-courses/ID511001-programming-2}{GitHub} for important course material, including changes to class scheduling or assessment details. Not checking will not be accepted as an excuse.

\subsection*{Snow Days/Polytechnic Closure}
In the event \textbf{Otago Polytechnic | Te Kura Matatini ki Otago} is closed or has a delayed opening because of snow or bad weather, you should not attempt to attend class if it is unsafe to do so. It is possible that the teaching staff will not be able to attend either, so classes will not physically be meeting. However, this does not become a holiday. Rather, the course material will be made available on \href{https://github.com/otago-polytechnic-bit-courses/ID511001-programming-2}{GitHub} for classes affected by the closure. You are responsible for any course material presented in this manner. Information about closure will be posted on the \textbf{Otago Polytechnic | Te Kura Matatini ki Otago Facebook} page \href{https://www.facebook.com/OtagoPoly}{https://www.facebook.com/OtagoPoly}.

\subsection*{Group Work \& Originality}
Learners in the \textbf{Bachelor of Information Technology} programme are expected to hand in original work. Learners are encouraged to discuss assessments with their fellow learners, however, all assessments are to be completed as individual works unless group work is explicitly required (i.e. if it doesn't say it is group work then it is not group work - even if a group consultation was involved). Failure to submit your original work will be treated as plagiarism.

\subsection*{Referencing}
Appropriate referencing is required for all work. Referencing standards will be specified by the teaching staff.

\subsection*{Plagiarism}
Plagiarism is submitting someone elses work as your own. Plagiarism offences are taken seriously \& an assessment that has been plagiarised may be awarded a zero mark. A definition of plagiarism is in the Student Handbook, available online or at the school office.

\subsection*{Submission Requirements}
All assessments are to be submitted by the time, date, \& method given when the assessment is issued. Failure to meet all requirements will result in a penalty of up to \textbf{10\%} per day (including weekends).

\subsection*{Extensions}
Extensions are only available for unusual circumstances. These must be applied for, \& approved, before the submission date.

\subsection*{Impairment}
In case of sickness contact the teaching staff or \textbf{Head of Information Technology (Michael Holtz)} as soon as possible, preferably before the assessment is due. The policy regarding the granting of a mark that considers impaired performance requires a medical certificate \& a medical practitioner’s signature on a form. You may refer to the guide on impaired performance on the student handbook.

\subsection*{Appeals}
If you are concerned about any aspect of your assessment, approach the teaching staff in the first instance. We support an open-door policy \& aim to resolve issues promptly. Further support is available from the \textbf{Head of Information Technology (Michael Holtz)} \& \textbf{First-Year Coordinator (Elise Allen)}. \textbf{Otago Polytechnic | Te Kura Matatini ki Otago} has a formal process for academic appeals if necessary.

\subsection*{Other Documents}
Regulatory documents relating to this course can be found on the \textbf{Otago Polytechnic | Te Kura Matatini ki Otago} website.

\end{document}