% Author: Grayson Orr
% Course: ID511001: Programming 2

\documentclass{article}
\author{}

\usepackage{graphicx}
\usepackage{wrapfig}
\usepackage{enumerate}
\usepackage{hyperref}
\usepackage[margin = 1.5cm]{geometry}
\usepackage[table]{xcolor}
\hypersetup{
  colorlinks = true,
  urlcolor = blue
}
\setlength\parindent{0pt}

\begin{document}

\begin{figure}
	\centering
	\includegraphics[width=50mm]{../img/logo.png}
\end{figure}

\title{Course Directive\\ID511001: Programming 2\\Semester One, 2023}
\date{}
\maketitle

\section*{Course Information}
\begin{tabular}{ll}
	Level:        & 5 \\
	Credits:      & 15                                                             \\
	Prerequisite: & ID510001: Programming 1                                  \\
	Timetable:    & Stream L: Monday 3 PM D202 \& Wednesday 8 AM D207  \\
	Lecturer-Led Tutorials: & Monday \& Thursday 8.30 PM - 10.00 PM Online     \\     
	Learner-Led Tutorials: & Tuesday \& Wednesday 12.00 PM - 1.00 PM D202    \\
\end{tabular}

\section*{Teaching Staff} 
\begin{tabular}{ll}
	Name:            & Grayson Orr                           \\
	Position:        & Senior Lecturer \& Second/Third-Year Coordinator \\
	Office Location: & D318                                 \\
	Email Address    & grayson.orr@op.ac.nz                    \\
\end{tabular}

\section*{Course Dates} 
\begin{tabular}{ll}
	Term 1:             &  Monday 20 February - Thursday 06 April \\
	Mid Semester Break: &  Friday 07 April - Friday 21 April     \\
	Term 2:             &  Monday 24 April - Friday 23 June      \\
\end{tabular}

\section*{Public Holidays \& Anniversary Days}
A list of public holidays \& anniversary days can be found here - \href{https://www.op.ac.nz/students/importantdates}{https://www.op.ac.nz/students/importantdates}

\section*{Aims}
To enable learners to build simple object-oriented (OO) applications and to identify situations that are most appropriate for OO implementation.

\section*{Learning Outcome}
At the successful completion of this course, learners will be able to:
\begin{enumerate}
	\item Build interactive, event-driven GUI applications using pre-built components.
	\item Declare \& implement user-defined classes using encapsulation, inheritance \& polymorphism.
\end{enumerate}

\section*{Assessments}
\renewcommand{\arraystretch}{1.5}
\begin{tabular}{|c|c|c|c|}
	\hline
	\textbf{Assessment}                                 & \textbf{Weighting} & \textbf{Due Date}            & \textbf{Learning Outcomes} \\ \hline
	\small Project 1 (C\# Console App): Learner Gradebook  & \small 25\%        & \small 26-04-2023 (Wednesday at 4.59 PM)   & \small 1 \& 2                   \\ \hline
	\small Project 2 (C\# Windows Forms App): Pong & \small 35\%        & \small 14-06-2023 (Wednesday at 4.59 PM)   & \small 1 \& 2                   \\ \hline
	\small Theory Examination                        & \small 30\%        & \small 21-06-2023 (Wednesday at 4.45 PM)  & \small 1 \& 2                   \\ \hline
	\small Classroom Tasks                       & \small 10\%        & \small 07-06-2023 (Wednesday at 4.59 PM)  & \small 1 \& 2                   \\ \hline
\end{tabular} 

\section*{Provisional Schedule}
\renewcommand{\arraystretch}{1.5}
\begin{tabular}{|c|c|c|c|}
	\hline
	\textbf{Week}                  & \textbf{Date Starting}            & \multicolumn{2}{c|}{\textbf{Topics}}                                                                                             \\ \hline
	\footnotesize 1/Tahi           & \footnotesize 20-02-2023 & \multicolumn{2}{c|}{\footnotesize GitHub Workflow \& C\# Basics}    \\ \hline
	\footnotesize 2/Rua            & \footnotesize 27-02-2023 & \multicolumn{2}{c|}{\footnotesize Classes \& Objects, \& Encapsulation}                   \\ \hline
	\footnotesize 3/Toru           & \footnotesize 06-03-2023 & \multicolumn{2}{c|}{\footnotesize Abstract Data Types} \\ \hline
	\footnotesize 4/Whā            & \footnotesize 13-03-2023 & \multicolumn{2}{c|}{\footnotesize Debugging \& Unit Testing}                               \\ \hline
	\footnotesize 5/Rima           & \footnotesize 20-03-2023 & \multicolumn{2}{c|}{\footnotesize Inheritance}                                                \\ \hline
	\footnotesize 6/Ono            & \footnotesize 27-03-2023  & \multicolumn{2}{c|}{\footnotesize Polymorphism \& Enumerations}                                                   \\ \hline
	\footnotesize 7/Whitu          & \footnotesize 03-04-2023 &  \multicolumn{2}{c|}{\footnotesize Project 1 Work}                            \\ \hline
	\rowcolor{yellow} \multicolumn{4}{|c|}{\footnotesize Mid Term Break}                                                                                                                         \\ \hline
	\footnotesize 8/Waru           & \footnotesize 24-04-2023 & \multicolumn{2}{c|}{\footnotesize Project 1 Work}                                                   \\ \hline

	\footnotesize 9/Iwa            & \footnotesize 01-05-2023 & \multicolumn{2}{c|}{\footnotesize Introduction to C\# Windows Forms Apps}                                                                 \\ \hline
	\footnotesize 10/Tekau         & \footnotesize 08-05-2023 & \multicolumn{2}{c|}{\footnotesize Timer Control \& Graphics Class}                                                                 \\ \hline
	\footnotesize 11/Tekau mā tahi & \footnotesize 15-05-2023 & \multicolumn{2}{c|}{\footnotesize More Graphics Class \& Controller Class}                                                                 \\ \hline
	\footnotesize 12/Tekau mā rua  & \footnotesize 22-05-2023 & \multicolumn{2}{c|}{\footnotesize More Controller Class}                                                                 \\ \hline
	\footnotesize 13/Tekau mā toru & \footnotesize 29-05-2023 & \multicolumn{2}{c|}{\footnotesize Project 2 Work}                                                     \\ \hline
	\footnotesize 14/Tekau mā whā  & \footnotesize 05-06-2023 & \multicolumn{2}{c|}{\footnotesize Project 2 Work} \\ \hline 
	\footnotesize 15/Tekau mā rima & \footnotesize 12-06-2023 & \multicolumn{2}{c|}{\footnotesize Theory Examination Preparation}                                                       \\ \hline
	\footnotesize 16/Tekau mā ono  & \footnotesize 19-06-2023 & \multicolumn{2}{c|}{\footnotesize Theory Examination}                                                         \\ \hline
\end{tabular}

\section*{Resources}

\subsection*{Software}
This paper will be taught using \textbf{Microsoft Visual Studio}. An installer for \textbf{Microsoft Visual Studio} is available - \href{https://visualstudio.microsoft.com/downloads}{https://visualstudio.microsoft.com/downloads}. Please refer any problems with downloads or installers to \textbf{Rob Broadley} in D205a.

\subsection*{Readings}
No textbook is required for this course. URLs to useful resources will be provided in the lecture notes.

\section*{Course Requirements \& Expectations}

\subsection*{Learning Hours}
This course requires \textbf{150 hours} of learning. This time includes \textbf{64 hours} of timetabled class time, \& \textbf{86 hours} of self-directed reading, preparation \& completion of assessments.

\subsection*{Criteria for Passing}
To pass this paper, you must achieve a cumulative pass mark of \textbf{50\%} over all assessments. There are no reassessments or resits. 

\subsection*{Attendance}
\begin{itemize}
	\item Learners are expected to attend all classes, including lectures \& labs.
	\item If you cannot attend for a few days for any reason, contact the course.
\end{itemize} 

\subsection*{Communication}
\textbf{Microsoft Outlook/Teams} are the official communication channels for this course. It is your responsibility to regularly check \textbf{Microsoft Outlook/Teams} \& \href{https://github.com/otago-polytechnic-bit-courses/ID511001-programming-2}{GitHub} for important course material, including changes to class scheduling or assessment details. Not checking will not be accepted as an excuse.

\subsection*{Snow Days/Polytechnic Closure}
In the event \textbf{Te Pūkenga} is closed or has a delayed opening because of snow or bad weather, you should not attempt to attend class if it is unsafe to do so. It is possible that the teaching staff will not be able to attend either, so classes will not physically be meeting. However, this does not become a holiday. Rather, the course material will be made available on \href{https://github.com/otago-polytechnic-bit-courses/ID511001-programming-2}{GitHub} for classes affected by the closure. You are responsible for any course material presented in this manner. Information about closure will be posted on the \textbf{Te Pūkenga Facebook} page \href{https://www.facebook.com/OtagoPoly}{https://www.facebook.com/OtagoPoly}.

\subsection*{Group Work \& Originality}
Learners in the \textbf{Bachelor of Information Technology} programme are expected to hand in original work. Learners are encouraged to discuss assessments with their fellow learners, however, all assessments are to be completed as individual works unless group work is explicitly required (i.e. if it doesn't say it is group work then it is not group work - even if a group consultation was involved). Failure to submit your original work will be treated as plagiarism.

\subsection*{ChatGPT}
In this course, you will be encouraged to use \textbf{ChatGPT} for your assessments. Learning to use Artificial Intelligence tools is an important skill. While \textbf{ChatGPT} is a powerful tool, you must be aware of the following:

\begin{itemize}
    \item If you provide \textbf{ChatGPT} with a prompt that is not refined enough, it may generate a not-so-useful response
    \item Do not trust \textbf{ChatGPT's} responses blindly. You must still use your judgement and may need to do additional research to determine if the response is correct
    \item Acknowledge that you are using \textbf{ChatGPT}. In the assessment's repository \textbf{README.md} file, please include what prompt(s) you provided to \textbf{ChatGPT} \& how you used the response(s) to help you with your work
\end{itemize}

\subsection*{Referencing}
Appropriate referencing is required for all work. Referencing standards will be specified by the teaching staff.

\subsection*{Plagiarism}
Plagiarism is submitting someone elses work as your own. Plagiarism offences are taken seriously \& an assessment that has been plagiarised may be awarded a zero mark. A definition of plagiarism is in the Student Handbook, available online or at the school office.

\subsection*{Submission Requirements}
All assessments are to be submitted by the time, date, \& method given when the assessment is issued. Failure to meet all requirements will result in a penalty of up to \textbf{10\%} per day (including weekends).

\subsection*{Extensions}
Extensions are only available for unusual circumstances. These must be applied for, \& approved, before the submission date.

\subsection*{Impairment}
In case of sickness contact the teaching staff or \textbf{Head of Information Technology (Michael Holtz)} as soon as possible, preferably before the assessment is due. The policy regarding the granting of a mark that considers impaired performance requires a medical certificate \& a medical practitioner’s signature on a form. You may refer to the guide on impaired performance on the student handbook.

\subsection*{Appeals}
If you are concerned about any aspect of your assessment, approach the teaching staff in the first instance. We support an open-door policy \& aim to resolve issues promptly. Further support is available from the \textbf{Head of Information Technology (Michael Holtz)} \& \textbf{First-Year Coordinator (Elise Allen)}. \textbf{Te Pūkenga} has a formal process for academic appeals if necessary.

\subsection*{Other Documents}
Regulatory documents relating to this course can be found on the \textbf{Te Pūkenga} website.

\end{document}