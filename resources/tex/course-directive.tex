% Author: Grayson Orr
% Course: ID511001: Programming 2

\documentclass{article}
\author{}
 
\usepackage{fontspec}
\setmainfont{Arial}

\usepackage{graphicx}
\usepackage{wrapfig}
\usepackage{enumerate}
\usepackage{hyperref}
\usepackage[margin = 1.5cm]{geometry}
\usepackage[table]{xcolor}
\hypersetup{
  colorlinks = true,
  urlcolor = blue
}
\setlength\parindent{0pt}

\begin{document}

\begin{figure}
	\centering
	\includegraphics[width=50mm]{../img/logo.png}
\end{figure}

\title{Course Directive\\ID511001: Programming 2\\Semester One, 2024}
\date{}
\maketitle

\section*{Course Information}
\begin{tabular}{ll}
	Level:        & 5 \\
	Credits:      & 15                                                             \\
	Prerequisite: & ID510001: Programming 1                                  \\
	Rōpū Mangu Timetable:    & Monday 3.00 PM D207 and Wednesday 1.00 PM D105b  \\
\end{tabular}

\section*{Teaching Staff} 
\begin{tabular}{lll}
	Name:             & Rachel Trounson & Grayson Orr                       \\
	Position:        & Senior Lecturer  & Senior Lecturer and Second/Third-Year Coordinator \\
	Office Location: & D316a  &          D309                         \\
	Email Address    &  rachel.trounson@op.ac.nz  &  grayson.orr@op.ac.nz                \\
\end{tabular} 

\section*{Course Dates}
\begin{tabular}{ll}
	Term 1:           & Monday 26 February - Friday 12 April  \\
	Mid Semester Break: &  Monday 15 April  - Friday 26 April     \\
	Term 2:             & Monday 29 April - Thursday 27 June       \\
\end{tabular}

\section*{Public Holidays and Anniversary Days}
A list of public holidays and anniversary days can be found here - \href{https://www.op.ac.nz/students/importantdates}{https://www.op.ac.nz/students/importantdates}

\section*{Aims}
To enable learners to build simple object-oriented (OO) applications and to identify situations that are most appropriate for OO implementation.

\section*{Learning Outcome}
At the successful completion of this course, learners will be able to:
\begin{enumerate}
	\item Build interactive, event-driven GUI applications using pre-built components.
	\item Declare and implement user-defined classes using encapsulation, inheritance and polymorphism.
\end{enumerate}

\section*{Assessments}
\renewcommand{\arraystretch}{1.5}
\begin{tabular}{|c|c|c|c|}
	\hline
	\textbf{Assessment}                                 & \textbf{Weighting} & \textbf{Due Date}            & \textbf{Learning Outcomes} \\ \hline
	\small Project 1 & \small 25\%        & \small 21-06-2024 (Friday at 4.59 PM)  & \small 1 and 2                   \\ \hline
	\small Project 2  & \small 35\%        & \small 03-05-2024 (Friday at 4.59 PM)   & \small 1 and 2                   \\ \hline
	\small Theory Examination                        & \small 30\%        & \small 26-06-2024 (Wednesday at 3.00 PM)  & \small 1 and 2                   \\ \hline
	\small Classroom Tasks                      & \small 10\%        & \small 03-05-2024 (Friday at 4.59 PM)  & \small 1 and 2                   \\ \hline
\end{tabular} 

\section*{Grade Table - Criterion Referenced}
\renewcommand{\arraystretch}{1.5}
\begin{tabular}{|c|c|}
	\hline
	\textbf{Grade} & \textbf{Mark Range} \\ \hline
	A+             & Met all course requirements-mark in range [90-100]           \\ \hline
	A              & Met all course requirements-mark in range [85-89]              \\ \hline
	A-             & Met all course requirements-mark in range [80-84]             \\ \hline
	B+             & Met all course requirements-mark in range [75-79]             \\ \hline
	B              & Met all course requirements-mark in range [70-74]             \\ \hline
	B-             & Met all course requirements-mark in range [65-69]             \\ \hline
	C+             & Met all course requirements-mark in range [60-64]             \\ \hline
	C              & Met all course requirements-mark in range [55-59]             \\ \hline
	C-             & Met all course requirements-mark in range [50-54]             \\ \hline
	D              & There at end. Did not meet course requirements. Mark in range [40-49]             \\ \hline
	E              & There at end. Did not meet course requirements. Mark in range [0-39]             \\ \hline
\end{tabular}

\section*{Provisional Schedule}
\renewcommand{\arraystretch}{1.5}
\begin{tabular}{|c|c|c|c|}
	\hline
	\textbf{Week}                  & \textbf{Date Starting}            & \multicolumn{2}{c|}{\textbf{Topics}}                                                                                             \\ \hline
	1/Tahi           & 26-02-2024 & \multicolumn{2}{c|}{GitHub and C\#}    \\ \hline
	2/Rua            & 04-03-2024 & \multicolumn{2}{c|}{Lists and LINQ}                   \\ \hline
	3/Toru           & 11-03-2024 & \multicolumn{2}{c|}{Windows Forms Application} \\ \hline
	4/Whā            & 18-03-2024 & \multicolumn{2}{c|}{Classes, Objects and Encapsulation}                               \\ \hline
	5/Rima           & 25-03-2024 & \multicolumn{2}{c|}{Inheritance and Polymorphism}                                                \\ \hline
	6/Ono            & 01-04-2024  & \multicolumn{2}{c|}{Interfaces and Enumerations}                                                   \\ \hline
	7/Whitu          & 08-04-2024 &  \multicolumn{2}{c|}{Debugging and Unit Testing}                            \\ \hline

	\rowcolor{yellow} \multicolumn{4}{|c|}{Mid Term Break}                                                                                                                         \\ \hline
	8/Waru   & 29-04-2024 & \multicolumn{2}{c|}{Assessment Work}                                                   \\ \hline
	9/Iwa            & 06-05-2024 & \multicolumn{2}{c|}{Timer Control, Graphics Class and SoundPlayer Class}                                                                 \\ \hline
	10/Tekau         & 13-05-2024 & \multicolumn{2}{c|}{More Graphics Class}                                                                 \\ \hline
	11/Tekau mā tahi & 20-05-2024 & \multicolumn{2}{c|}{Key Events}                                                                 \\ \hline
	12/Tekau mā rua  & 27-05-2024 & \multicolumn{2}{c|}{Assessment Work}                                                                 \\ \hline
	13/Tekau mā toru & 03-06-2024 & \multicolumn{2}{c|}{Assessment Work}                                                     \\ \hline
	14/Tekau mā whā  & 10-06-2024 & \multicolumn{2}{c|}{Assessment Work} \\ \hline 
	15/Tekau mā rima & 17-06-2024 & \multicolumn{2}{c|}{Theory Examination Preparation}                                                       \\ \hline
	16/Tekau mā ono  & 24-06-2024 & \multicolumn{2}{c|}{Theory Examination}                                                         \\ \hline
\end{tabular}

\section*{Resources}

\subsection*{Software}
This paper will be taught using \textbf{Microsoft Visual Studio}. An installer for \textbf{Microsoft Visual Studio} is available - \href{https://visualstudio.microsoft.com/downloads}{https://visualstudio.microsoft.com/downloads}. Please refer any problems with downloads or installers to \textbf{Rob Broadley} in D205a.

\subsection*{Readings}
No textbook is required for this course. URLs to useful resources will be provided in the lecture notes.

\section*{Course Requirements and Expectations}

\subsection*{Learning Hours}
This course requires \textbf{150 hours} of learning. This time includes \textbf{60 hours} directed learning hours and \textbf{90} self-directed learning hours.

\subsection*{Criteria for Passing}
To pass this paper, you \textbf{must} achieve a cumulative pass mark of \textbf{50\%} over all assessments. There are no reassessments or resits. 

\subsection*{Attendance}
\begin{itemize}
	\item Learners are expected to attend all classes, including lectures and labs.
	\item If you cannot attend for a few days for any reason, contact the course.
\end{itemize} 

\subsection*{Communication}
\textbf{Microsoft Outlook/Teams} are the official communication channels for this course. It is your responsibility to regularly check \textbf{Microsoft Outlook/Teams} and \href{https://github.com/otago-polytechnic-bit-courses/ID511001-programming-2}{GitHub} for important course material, including changes to class scheduling or assessment details. Not checking will not be accepted as an excuse.

\subsection*{Snow Days/Polytechnic Closure}
In the event \textbf{Otago Polytechnic | Te Pūkenga} is closed or has a delayed opening because of snow or bad weather, you should not attempt to attend class if it is unsafe to do so. It is possible that the teaching staff will not be able to attend either, so classes will not physically be meeting. However, this does not become a holiday. Rather, the course material will be made available on \href{https://github.com/otago-polytechnic-bit-courses/ID511001-programming-2}{GitHub} for classes affected by the closure. You are responsible for any course material presented in this manner. Information about closure will be posted on the \textbf{Otago Polytechnic | Te Pūkenga Facebook} page \href{https://www.facebook.com/OtagoPoly}{https://www.facebook.com/OtagoPoly}.

\subsection*{Group Work and Originality}
Learners in the \textbf{Bachelor of Information Technology} programme are expected to hand in original work. Learners are encouraged to discuss assessments with their fellow learners, however, all assessments are to be completed as individual works unless group work is explicitly required (i.e. if it doesn't say it is group work then it is not group work - even if a group consultation was involved). Failure to submit your original work will be treated as plagiarism.

\subsection*{ChatGPT}
Learning to use \textbf{Artificial Intelligence tools} like \textbf{ChatGPT} is an important skill. While \textbf{ChatGPT} is a powerful tool, you \textbf{must} be aware of the following:

\begin{itemize}
    \item If you provide \textbf{ChatGPT} with a prompt that is not refined enough, it may generate a not-so-useful response
    \item Do not trust \textbf{ChatGPT's} responses blindly. You \textbf{must} still use your judgement and may need to do additional research to determine if the response is correct
    \item Acknowledge that you are using \textbf{ChatGPT}. In the assessment's repository \textbf{README.md} file, please include what prompt(s) you provided to \textbf{ChatGPT} and how you used the response(s) to help you with your work
\end{itemize}

\subsection*{Referencing}
Appropriate referencing is required for all work. Referencing standards will be specified by the teaching staff.

\subsection*{Plagiarism}
Plagiarism is submitting someone elses work as your own. Plagiarism offences are taken seriously and an assessment that has been plagiarised may be awarded a zero mark. A definition of plagiarism is in the Student Handbook, available online or at the school office.

\subsection*{Submission Requirements}
All assessments are to be submitted by the time, date, and method given when the assessment is issued. Failure to meet all requirements will result in a penalty of up to \textbf{10\%} per day (including weekends).

\subsection*{Extensions}
Familiarise yourself with the assessment due dates. Extensions will \textbf{only} be granted if you are unable to complete the assessment by the due date because of \textbf{unforeseen circumstances outside your control}. The length of the extension granted will depend on the circumstances and \textbf{must} be negotiated with the course lecturer before the assessment due date. A medical certificate or support letter may be needed. Extensions will not be granted for poor time management or pressure of other assessments.

\subsection*{Impairment}
In case of sickness contact the teaching staff or \textbf{Head of Information Technology (Michael Holtz)} as soon as possible, preferably before the assessment is due. The policy regarding the granting of a mark that considers impaired performance requires a medical certificate and a medical practitioner’s signature on a form. You may refer to the guide on impaired performance on the student handbook.

\subsection*{Appeals}
If you are concerned about any aspect of your assessment, approach the teaching staff in the first instance. We support an open-door policy and aim to resolve issues promptly. Further support is available from the \textbf{Head of Information Technology (Michael Holtz)} and \textbf{First-Year Coordinator (Elise Allen)}. \textbf{Otago Polytechnic | Te Pūkenga} has a formal process for academic appeals if necessary.

\subsection*{Other Documents}
Regulatory documents relating to this course can be found on the \textbf{Otago Polytechnic | Te Pūkenga} website.

\end{document}