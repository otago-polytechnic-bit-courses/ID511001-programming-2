% Author: Grayson Orr
% Course: ID511001: Programming 2

\documentclass{article}
\author{}

\usepackage{graphicx}
\usepackage{wrapfig}
\usepackage{enumerate}
\usepackage{hyperref}
\usepackage[margin = 2.25cm]{geometry}
\usepackage[table]{xcolor}
\usepackage{soul}
\usepackage{fancyhdr}
\usepackage{fancyvrb}
\hypersetup{
  colorlinks = true,
  urlcolor = blue
}
\setlength\parindent{0pt}
\pagestyle{fancy}
\fancyhf{}
\rhead{College of Engineering, Construction and Living Sciences\\Bachelor of Information Technology}
\lfoot{Project 1\\Version 3, Semester One, 2024}
\rfoot{\thepage}
 
\begin{document}

\begin{figure}
    \centering
    \includegraphics[width=50mm]{../../resources/img/logo.png}
\end{figure}

\title{College of Engineering, Construction and Living Sciences\\Bachelor of Information Technology\\ID511001: Programming 2\\Level 5, Credits 15\\\textbf{Project 1}}
\date{}
\maketitle

\section*{Assessment Overview}
In this \textbf{individual} assessment, you will design and develop a pong \textbf{Windows Forms App} using \textbf{C\#}.

\section*{Learning Outcomes}
At the successful completion of this course, learners will be able to:
\begin{enumerate}
    \item Build interactive, event-driven GUI applications using pre-built components.
    \item Declare and implement user-defined classes using encapsulation, inheritance and polymorphism.
\end{enumerate} 

\section*{Assessments}
\renewcommand{\arraystretch}{1.5}
\begin{tabular}{|c|c|c|c|}
	\hline
	\textbf{Assessment}                                 & \textbf{Weighting} & \textbf{Due Date}            & \textbf{Learning Outcomes} \\ \hline
	\small Project 1 & \small 25\%        & \small 24-06-2024 (Monday at 4.59 PM)  & \small 1 and 2                   \\ \hline
	\small Project 2  & \small 35\%        & \small 29-04-2024 (Monday at 4.59 PM)   & \small 1 and 2                   \\ \hline
	\small Theory Examination                        & \small 30\%        & \small 26-06-2024 (Wednesday at 3.00 PM)  & \small 1 and 2                   \\ \hline
	\small Classroom Tasks                      & \small 10\%        & \small 29-04-2024 (Monday at 4.59 PM)  & \small 1 and 2                   \\ \hline
\end{tabular} 

\section*{Conditions of Assessment}
You will complete this assessment during your learner-managed time. However, there will be time during class to discuss the requirements and your progress on this assessment. This assessment will need to be completed by \textbf{Monday, 29 April 2024} at \textbf{4.59 PM}.

\section*{Pass Criteria}
This assessment is criterion-referenced (CRA) with a cumulative pass mark of \textbf{50\%} over all assessments in \textbf{ID511001: Programming 2}.

\section*{Authenticity}
All parts of your submitted assessment \textbf{must} be completely your work. Do your best to complete this assessment without using an \textbf{AI generative tool}. You need to demonstrate to the course lecturer that you can meet the learning outcome(s) for this assessment. \\
 
 However, if you get stuck, you can use an \textbf{AI generative tool} to help you get unstuck, permitting you to acknowledge that you have used it. In the assessment's repository \textbf{README.md} file, please include what prompt(s) you provided to the \textbf{AI generative tool} and how you used the response(s) to help you with your work. It also applies to code snippets retrieved from \textbf{StackOverflow} and \textbf{GitHub}. \\
 
 Failure to do this may result in a mark of \textbf{zero} for this assessment.

\section*{Policy on Submissions, Extensions, Resubmissions and Resits}
The school's process concerning submissions, extensions, resubmissions and resits complies with \textbf{Otago Polytechnic | Te Pūkenga} policies. Learners can view policies on the \textbf{Otago Polytechnic | Te Pūkenga} website located at \href{https://www.op.ac.nz/about-us/governance-and-management/policies}{https://www.op.ac.nz/about-us/governance-and-management/policies}.

\section*{Submission}
You \textbf{must} submit all application files via \textbf{GitHub Classroom}. Here is the URL to the repository you will use for your submission – \href{https://classroom.github.com/a/Wx7UHym1}{https://classroom.github.com/a/Wx7UHym1}. If you do not have not one, create a \textbf{.gitignore} and add the ignored files in this resource - \href{https://raw.githubusercontent.com/github/gitignore/main/VisualStudio.gitignore}{https://raw.githubusercontent.com/github/gitignore/main/VisualStudio.gitignore}. Create a branch called \textbf{project-1}. The latest application files in the \textbf{project-1} branch will be used to mark against the \textbf{Functionality} criterion. Please test before you submit. Partial marks \textbf{will not} be given for incomplete functionality. Late submissions will incur a \textbf{10\% penalty per day}, rolling over at \textbf{5:00 PM}.

\section*{Extensions}
Familiarise yourself with the assessment due date. Extensions will \textbf{only} be granted if you are unable to complete the assessment by the due date because of \textbf{unforeseen circumstances outside your control}. The length of the extension granted will depend on the circumstances and \textbf{must} be negotiated with the course lecturer before the assessment due date. A medical certificate or support letter may be needed. Extensions will not be granted for poor time management or pressure of other assessments.

\section*{Resits}
Resits and reassessments \textbf{are not} applicable in \textbf{ID511001: Programming 2}.

\section*{Instructions}
You will need to submit an application and documentation that meet the following requirements:\\

\subsection*{Functionality - Learning Outcomes 1 and 2 (50\%)}
\begin{itemize}
    \item The application needs to open without code or file structure modification in \textbf{Visual Studio}.
    \item The game needs to be driven by one \textbf{Timer} and begins when the user presses the \textbf{space bar} key.
    \item The ball and two paddles need to be created using the \textbf{Graphics} class. 
    \item The ball needs to collide off the top and bottom of the screen, and paddles.
    \item The paddles needs to move vertically but not exceed the top and bottom of the screen.
    \item The user controls the left paddle via the \textbf{up arrow} and \textbf{down arrow} keys. The computer controls the right paddle. It is acceptable for the right paddle to follow the ball's position. However, other solutions are encouraged.
    \item Display the user and computer's score using the \textbf{DrawString} method.
    \item A scoring system. When the ball collides with the left and right-hand side of the screen, one point is given to either the user or computer. The game is over when either score is 10. 
    \item A highscore system. When the game is over, appropriate feedback needs to be displayed to the user, i.e., \textbf{"You win!"} or \textbf{"You lose!"}, the user and computer's scores are saved, i.e., written to a text file. Read the scores from the text file and display the last five to the user.
    \item Double buffering to prevent the ball, paddles and scores from flickering.
    \item Using the \textbf{SoundPlayer} class, play a sound when:
    \begin{itemize}
        \item The ball bounces off the paddle, and top and bottom of the screen.
        \item The user wins and loses.
    \end{itemize}
    \item Restart and pause a game via the \textbf{R} and \textbf{P} keys.
    \item Randomise the colour of the ball and paddles.
\end{itemize}

\subsection*{Code Elegance - Learning Outcomes 1 and 2 (40\%)}
\begin{itemize}
    \item A \textbf{Visual Studio} \textbf{.gitignore} file is used. 
    \item Appropriate naming of files, variables, methods and classes.
    \item Idiomatic use of values, control flow, data structures and in-built functions.
    \item Efficient algorithmic approach.
    \item Sufficient modularity.
    \item Each file has an \textbf{XML documentation comment} located at the top of the file.
    \item \textbf{XML documentation comments} where required. It should be for code that needs further explanation.
    \item Formatted code.
    \item No dead or unused code.
\end{itemize}

\subsection*{Documentation and Git Usage - Learning Outcomes 1 and 2 (10\%)}
\begin{itemize}
    \item Provide the following in your repository \textbf{README.md} file:
    \begin{itemize}
        \item A class diagram of your application.
        \item If applicable, known bugs.
    \end{itemize}
    \item Commit messages reflect the context of each functional requirement change.
\end{itemize}

\subsection*{Additional Information}
\begin{itemize}
    \item An exemplar is available in the \textbf{assessment} directory of the \textbf{course materials} repository.
    \item You may add additional classes and methods.
    \item When the user presses a key, i.e., \textbf{up arrow} or \textbf{down arrow}, a \textbf{KeyDown} event is generated. For the \textbf{Form1's} \textbf{KeyDown} event, the method signature is:
    \begin{Verbatim}[tabsize=2]
		private void Form1_KeyDown(object sender, KeyEventArgs e) {}
	\end{Verbatim}
	The argument you will be interested in is \textbf{KeyEventArgs e} which is the value of the pressed key. The \textbf{arrow} key values are \textbf{Keys.Left}, \textbf{Keys.Right}, \textbf{Keys.Up} and \textbf{Keys.Down}. In the \textbf{Form1\_KeyDown} method, you can use a \textbf{switch} statement. For example:
	\begin{Verbatim}[tabsize=2]
		switch (e.KeyCode) 
		{
			case Keys.Left:
				// Do something
				break;
			case Keys.Right:
				// Do something
				break;
			case Keys.Up:
				// Do something
				break;
			case Keys.Down:
				// Do something
				break;
			default:
				// Do something
				break;
		}
	\end{Verbatim}
	\textbf{Note:} The \textbf{Form1's} \textbf{KeyPreview} event needs to be set to \textbf{True}. Otherwise, \textbf{Form1} will not respond to the \textbf{KeyDown} event.
    \item \textbf{Do not} rewrite your \textbf{Git} history. It is important that the course lecturer can see how you worked on your assessment over time
\end{itemize}

\end{document}