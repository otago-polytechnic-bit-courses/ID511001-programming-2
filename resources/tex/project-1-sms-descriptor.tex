% Author: Grayson Orr
% Course: ID511001: Programming 2

\documentclass{article}
\author{}

\usepackage{graphicx}
\usepackage{wrapfig}
\usepackage{enumerate}
\usepackage{hyperref}
\usepackage[margin = 2.25cm]{geometry}
\usepackage[table]{xcolor}
\usepackage{soul}
\usepackage{fancyhdr}
\usepackage{fancyvrb}
\hypersetup{
  colorlinks = true,
  urlcolor = blue
}
\setlength\parindent{0pt}
\pagestyle{fancy}
\fancyhf{}
\rhead{College of Engineering, Construction and Living Sciences\\Bachelor of Information Technology}
\lfoot{Project 1: Student Management System\\Version 2, Semester Two, 2023}
\rfoot{\thepage}
 
\begin{document}

\begin{figure}
    \centering
    \includegraphics[width=50mm]{../../resources/img/logo.png}
\end{figure}

\title{College of Engineering, Construction and Living Sciences\\Bachelor of Information Technology\\ID511001: Programming 2\\Level 5, Credits 15\\\textbf{Project 1: Student Management System}}
\date{}
\maketitle

\section*{Assessment Overview}
In this assessment, you will design and develop a student management system \textbf{Console Application} using \textbf{C\#}. 

\section*{Learning Outcomes}
At the successful completion of this course, learners will be able to:
\begin{enumerate}
    \item Build interactive, event-driven GUI applications using pre-built components.
    \item Declare and implement user-defined classes using encapsulation, inheritance and polymorphism.
\end{enumerate}

\section*{Assessments}
\renewcommand{\arraystretch}{1.5}
\begin{tabular}{|c|c|c|c|}
	\hline
	\textbf{Assessment}                                 & \textbf{Weighting} & \textbf{Due Date}            & \textbf{Learning Outcomes} \\ \hline
	\small Project 1: Student Management System  & \small 35\%        & \small 22-09-2023 (Friday at 4.59 PM)   & \small 1 and 2                   \\ \hline
	\small Project 2: Pong & \small 25\%        & \small 10-11-2023 (Friday at 04.59 PM)  & \small 1 and 2                   \\ \hline
	\small Theory Examination                        & \small 30\%        & \small 15-11-2023 (Wednesday at 12.10 PM)  & \small 1 and 2                   \\ \hline
	\small Classroom Task: Unit Testing                       & \small 10\%        & \small 22-09-2023 (Friday at 4.59 PM)  & \small 1 and 2                   \\ \hline
\end{tabular} 

\section*{Conditions of Assessment}
You will complete this assessment during your learner-managed time. However, there will be time during class to discuss the requirements and your progress on this assessment. This assessment will need to be completed by \textbf{Friday, 22 September 2023} at \textbf{4.59 PM}.

\section*{Pass Criteria}
This assessment is criterion-referenced (CRA) with a cumulative pass mark of \textbf{50\%} over all assessments in \textbf{ID511001: Programming 2}.

\section*{Authenticity}
All parts of your submitted assessment \textbf{must} be completely your work. If you use code snippets from \textbf{GitHub}, \textbf{StackOverflow} or other online resources, you \textbf{must} reference it appropriately using \textbf{APA 7th edition}. Provide your references in the \textbf{README.md} file in your repository. Failure to do this will result in a mark of \textbf{zero} for this assessment. 

\section*{Policy on Submissions, Extensions, Resubmissions and Resits}
The school's process concerning submissions, extensions, resubmissions and resits complies with \textbf{Otago Polytechnic | Te Pūkenga} policies. Learners can view policies on the \textbf{Otago Polytechnic | Te Pūkenga} website located at \href{https://www.op.ac.nz/about-us/governance-and-management/policies}{https://www.op.ac.nz/about-us/governance-and-management/policies}.

\section*{Submission}
You \textbf{must} submit all application files via \textbf{GitHub Classroom}. Here is the URL to the repository you will use for your submission – \href{https://classroom.github.com/a/xIHtZr71}{https://classroom.github.com/a/xIHtZr71}.  Create a \textbf{.gitignore} and add the ignored files in this resource - \href{https://raw.githubusercontent.com/github/gitignore/main/VisualStudio.gitignore}{https://raw.githubusercontent.com/github/gitignore/main/VisualStudio.gitignore}. The latest application files in the \textbf{master} or \textbf{main} branch will be used to mark against the \textbf{Functionality} criterion. Please test before you submit. Partial marks \textbf{will not} be given for incomplete functionality. Late submissions will incur a \textbf{10\% penalty per day}, rolling over at \textbf{5:00 PM}.

\section*{Extensions}
Familiarise yourself with the assessment due date. Extensions will \textbf{only} be granted if you are unable to complete the assessment by the due date because of \textbf{unforeseen circumstances outside your control}. The length of the extension granted will depend on the circumstances and \textbf{must} be negotiated with the course lecturer before the assessment due date. A medical certificate or support letter may be needed. Extensions will not be granted for poor time management or pressure of other assessments.

\section*{Resubmissions}
Learners may be requested to resubmit an assessment following a rework of part/s of the original assessment. Resubmissions are to be completed within a negotiable short time frame and usually \textbf{must} be completed within the timing of the course to which the assessment relates. Resubmissions will be available to learners who have made a genuine attempt at the first assessment opportunity and achieved a \textbf{D grade (40-49\%)}. The maximum grade awarded for resubmission will be \textbf{C-}.

\section*{Resits}
Resits and reassessments \textbf{are not} applicable in \textbf{ID511001: Programming 2}.

\section*{Instructions}
You will need to submit an application and documentation that meet the following requirements:\\

\subsection*{Functionality - Learning Outcomes 1 and 2 (50\%)}
\begin{itemize}
    \item The application needs to open without code or file structure modification in \textbf{Visual Studio}.
    \item The application needs to contain the following \textbf{enums}:
    \begin{verbatim}
        public enum Position
        {
            LECTURER,
            SENIOR_LECTURER,
            PRINCIPAL_LECTURER,
            ASSOCIATE_PROFESSOR,
            PROFESSOR
        }
            
        public enum Salary
        {
            LECTURER_SALARY = 85000,
            SENIOR_LECTURER_SALARY = 100000,
            PRINCIPAL_LECTURER_SALARY = 115000,
            ASSOCIATE_PROFESSOR_SALARY = 130000,
            PROFESSOR_SALARY = 145000
        }
    \end{verbatim}
    \item The application needs to contain the following \textbf{classes}:
    \begin{itemize}
        \item \textbf{Institution}. This \textbf{public} class has the following \textbf{private} fields: \textbf{name} of type \textbf{string}, \textbf{region} of type \textbf{string} and \textbf{country} of type \textbf{string}. 
        \item \textbf{Department}. This \textbf{public} class has the following \textbf{private} fields: \textbf{institution} of type \textbf{Institution} and \textbf{name} of type \textbf{string}.
        \item \textbf{Course}. This \textbf{public} class has the following \textbf{private} fields: \textbf{department} of type \textbf{Department}, \textbf{code} of type \textbf{string}, \textbf{name} of type \textbf{string}, \textbf{description} of type \textbf{string}, \textbf{credits} of type \textbf{int} and \textbf{fees} of type \textbf{double}.
        \item \textbf{CourseAssessmentMark}. This \textbf{public} class has the following \textbf{private} fields: \textbf{course} of type \textbf{Course} and \textbf{assessmentMarks} of type \textbf{List$<$int$>$}. Also, this \textbf{public} class has the following \textbf{public} methods:
        \begin{itemize}
            \item \textbf{GetAllMarks()} with the return type of \textbf{List$<$int$>$}. This method returns all assessment marks.
            \item \textbf{GetAllGrades()} with the return type of \textbf{List$<$string$>$}. This method returns all assessment grades.
            \item \textbf{GetHighestMarks()} with the return type of \textbf{int}. This method returns the highest passing assessment mark(s).
            \item \textbf{GetLowestMarks()} with the return type of \textbf{int}. This method returns the lowest passing assessment mark(s). 
            \item \textbf{GetFailMarks()} with the return type of \textbf{int}. This method returns the fail assessment mark(s).
            \item \textbf{GetAverageMarks()} with the return type of \textbf{double}. This method returns the average assessment mark.
            \item \textbf{GetAverageGrade()} with the return type of \textbf{string}. This method returns the average assessment grade.
        \end{itemize}
        For more information on how to calculate the highest, lowest and fail marks, refer to the \textbf{grade table} in the \textbf{Additional Information} section below.
        \item \textbf{Person}. This \textbf{parent} and \textbf{public} class has the following \textbf{protected} fields: \textbf{id} of type \textbf{int}, \textbf{firstName} of type \textbf{string} and \textbf{lastName} of type \textbf{string}.
        \item \textbf{Learner}. This \textbf{child} and \textbf{public} class inherits from \textbf{Person} and has additional one \textbf{private} field: \textbf{courseAssessmentMarks} of type \textbf{CourseAssessmentMark}.
        \item \textbf{Lecturer}. This \textbf{child} and \textbf{public} class inherits from \textbf{Person} and three additional \textbf{private} fields: \textbf{position} of type \textbf{Position}, \textbf{salary} of type \textbf{Salary} and \textbf{course} of type \textbf{Course}.
        \item \textbf{Utils}. This \textbf{static} class has the following \textbf{public static} methods:
        \begin{itemize}
            \item \textbf{SeedInstitutions()} with the return type of \textbf{List$<$Institution$>$}. This method seeds a \textbf{List$<$Institution$>$} with \textbf{three} \textbf{Institution} objects.
            \item \textbf{SeedDepartments()} with the return type of \textbf{List$<$Department$>$}. This method seeds a \textbf{List$<$Department$>$} with \textbf{three} \textbf{Department} objects.
            \item \textbf{SeedCourses()} with the return type of \textbf{List$<$Course$>$}. This method seeds a \textbf{List$<$Course$>$} with \textbf{three} \textbf{Course} objects. 
            \item \textbf{ReadLearnersFromFile()} with no return type and two parameters: \textbf{filePath} of type \textbf{string} and \textbf{learners} of type \textbf{List$<$Learner$>$}. This method reads the \textbf{learners.txt} file and populates the \textbf{learners} parameter. The \textbf{learners.txt} file contains the following information:
            \begin{verbatim}
                1,John,Doe,0,100,100,95,10,0
                2,Jane,Doe,0,45,35,45,75,65
                3,Grayson,Orr,1,50,60,75,80,55
                4,Joe,Blogs,1,10,20,30,70,80
                5,Bob,Brown,2,75,82,95,55,10
            \end{verbatim}
            \item \textbf{ReadLecturersFromFile()} with no return type and two parameters: \textbf{filePath} of type \textbf{string} and \textbf{lecturers} of type \textbf{List$<$Lecturer$>$}. This method reads the \textbf{lecturers.txt} file and populates the \textbf{lecturers} parameter. he \textbf{lecturers.txt} file contains the following information:
            \begin{verbatim}
                1,Graydon,Ore,1,100000,0
                2,Aidan,Moscow,2,115000,1
                3,Jon,Seena,0,85000,2
            \end{verbatim}
        \end{itemize}
        \textbf{Program}. This \textbf{public} class manages the business logic and user interface of the application. This class needs to account for the following functionality:
        \begin{itemize}
            \item Displaying course details. This needs to display the following details:
            \begin{itemize}
                \item course \textbf{code} and \textbf{name} in this format - \textbf{code: name}
                \item course \textbf{description}
                \item course \textbf{credits}
                \item course \textbf{fees}
                \item institution \textbf{name}, \textbf{region} and \textbf{country} in this format - \textbf{name, region, country}
                \item department \textbf{name}
            \end{itemize}
            \item Displaying all marks. This needs to display the following details:
            \begin{itemize}
                \item learner \textbf{id}
                \item learner \textbf{first name} and \textbf{last name} in this format - \textbf{first name last name}
                \item course \textbf{code} and \textbf{name} in this format - \textbf{code: name}
                \item learner \textbf{assessment marks 1-5} in this format - \textbf{mark 1, mark 2, mark 3, mark 4, mark 5}
            \end{itemize}
            \item Displaying all grades. This needs to display the following details:
            \begin{itemize}
                \item learner \textbf{id}
                \item learner \textbf{first name} and \textbf{last name} in this format - \textbf{first name last name}
                \item course \textbf{code} and \textbf{name} in this format - \textbf{code: name}
                \item learner \textbf{assessment grades 1-5} in this format - \textbf{grade 1, grade 2, grade 3, grade 4, grade 5}
            \end{itemize}
            \item Displaying highest marks. This needs to display the following details:
            \begin{itemize}
                \item learner \textbf{id}
                \item learner \textbf{first name} and \textbf{last name} in this format - \textbf{first name last name}
                \item course \textbf{code} and \textbf{name} in this format - \textbf{code: name}
                \item learner \textbf{highest assessment mark(s)}
            \end{itemize}
            \item Displaying lowest marks. This needs to display the following details:
            \begin{itemize}
                \item learner \textbf{id}
                \item learner \textbf{first name} and \textbf{last name} in this format - \textbf{first name last name}
                \item course \textbf{code} and \textbf{name} in this format - \textbf{code: name}
                \item learner \textbf{lowest assessment mark(s)}
            \end{itemize}
            \item Displaying fail marks. This needs to display the following details:
            \begin{itemize}
                \item learner \textbf{id}
                \item learner \textbf{first name} and \textbf{last name} in this format - \textbf{first name last name}
                \item course \textbf{code} and \textbf{name} in this format - \textbf{code: name}
                \item learner \textbf{fail assessment mark(s)}
            \end{itemize}
            \item Displaying average marks. This needs to display the following details:
            \begin{itemize}
                \item learner \textbf{id}
                \item learner \textbf{first name} and \textbf{last name} in this format - \textbf{first name last name}
                \item course \textbf{code} and \textbf{name} in this format - \textbf{code: name}
                \item learner \textbf{average assessment mark}
            \end{itemize}
            \item Displaying average grades. This needs to display the following details:
            \begin{itemize}
                \item learner \textbf{id}
                \item learner \textbf{first name} and \textbf{last name} in this format - \textbf{first name last name}
                \item course \textbf{code} and \textbf{name} in this format - \textbf{code: name}
                \item learner \textbf{average assessment grade}
            \end{itemize}
            \item Adding a learner. When adding a learner, the \textbf{id} is auto-generated and unique. Prompt the user to enter the following details: \textbf{first name}, \textbf{last name}, \textbf{course} and \textbf{assessment marks 1-5}. Implement the following validation:
            \begin{itemize}
                \item \textbf{first name} and \textbf{last name} are not empty or, contain numbers, or special characters.
                \item \textbf{course} is valid number.
                \item \textbf{assessment marks 1-5} are between 0 and 100.
            \end{itemize} 
            If the validation is successful, add the learner to the \textbf{learners} list, write the learner to the \textbf{learners.txt} file and display a success message. Otherwise, display an error message.             
            \item Removing a learner. When removing a learner, prompt the user to enter the \textbf{id} of the learner to remove. If the \textbf{id} is valid, prompt the user to confirm the removal. If the user confirms the removal, remove the learner from the \textbf{learners} list and \textbf{learners.txt} file, and display a success message. Otherwise, display an error message.             
            \item Displaying lecturer details. This needs to display the following details:
            \begin{itemize}
                \item lecturer \textbf{id}
                \item lecturer \textbf{first name} and \textbf{last name} in this format - \textbf{first name last name}
                \item lecturer \textbf{position}
                \item institution \textbf{name}, \textbf{region} and \textbf{country} in this format - \textbf{name, region, country}
                \item department \textbf{name}
                \item course \textbf{code} and \textbf{name} in this format - \textbf{code: name}
                \item lecturer \textbf{salary}
            \end{itemize}
            \item Adding a lecturer. When adding a lecturer, the \textbf{id} is auto-generated and unique. The \textbf{salary} is calculated based on the \textbf{position}. Prompt the user to enter the following details: \textbf{first name}, \textbf{last name}, \textbf{position} and \textbf{course}. Implement the following validation:
            \begin{itemize}
                \item \textbf{first name} and \textbf{last name} are not empty or, contain numbers, or special characters.
                \item \textbf{position} and \textbf{course} are valid numbers.
            \end{itemize}
            If the validation is successful, add the lecturer to the \textbf{lecturers} list, write the lecturer to the \textbf{lecturers.txt} file and display a success message. Otherwise, display an error message.
            \item Exiting the application
        \end{itemize} 
        If the user enters an invalid option, display an error message and prompt the user to enter a valid option.
    \end{itemize}
\end{itemize}

\subsection*{Code Elegance - Learning Outcomes 1 and 2 (40\%)}
\begin{itemize}
    \item A \textbf{Visual Studio} \textbf{.gitignore} file is used. 
    \item Appropriate naming of files, variables, methods and classes.
    \item Idiomatic use of control flow, data structures and in-built functions.
    \item Efficient algorithmic approach.
    \item Sufficient modularity.
    \item Each file has a header comment located at the top of the file.
    \item In-line comments where required. It should be for code that needs further explanation.
    \item Formatted code.
    \item No dead or unused code.
\end{itemize}

\subsection*{Documentation and Git Usage - Learning Outcomes 1 and 2 (10\%)}
\begin{itemize}
    \item Provide the following in your repository \textbf{README.md} file:
    \begin{itemize}
        \item A class diagram of your application.
        \item If applicable, known bugs.
    \end{itemize}
    \item Commit messages reflect the context of each functional requirement change.
\end{itemize}

\subsection*{Additional Information}
\begin{itemize}
    \item An exemplar is available in the \textbf{assessment} directory of the \textbf{course materials} repository.
    \item You may add additional classes and methods. 
    \item Grade and mark range table:\\\\
    \renewcommand{\arraystretch}{1.5}
    \begin{tabular}{|c|c|}
        \hline
        \textbf{Grade} & \textbf{Mark Range} \\ \hline
        A+ & 90-100  \\ \hline
        A & 85-89  \\ \hline
        A- & 80-84 \\ \hline
        B+ & 75-79   \\ \hline
        B & 70-74  \\ \hline
        B- & 65-69  \\ \hline
        C+ & 60-64  \\ \hline
        C & 55-59 \\ \hline
        C- (passing assessment marks) & 50-54  \\ \hline
        D & 40-49 (fail assessment marks)   \\ \hline
        E & 0-39 (fail assessment marks)   \\ \hline
    \end{tabular}
    \item \textbf{Do not} rewrite your \textbf{Git} history. It is important that the course lecturer can see how you worked on your assessment over time.
\end{itemize}

\end{document}