% Author: Grayson Orr
% Course: ID511001: Programming 2

\documentclass{article}
\author{}
 
\usepackage{fontspec}
\setmainfont{Arial}

\usepackage{graphicx}
\usepackage{wrapfig}
\usepackage{enumerate}
\usepackage{hyperref}
\usepackage[margin = 2.25cm]{geometry}
\usepackage[table]{xcolor}
\usepackage{soul}
\usepackage{fancyhdr}
\usepackage{fancyvrb}
\hypersetup{
  colorlinks = true,
  urlcolor = blue
}
\setlength\parindent{0pt}
\pagestyle{fancy}
\fancyhf{}
\rhead{College of Engineering, Construction and Living Sciences\\Bachelor of Information Technology}
\lfoot{Project 2\\Version 4, Semester Two, 2024}
\rfoot{\thepage}
 
\begin{document}

\begin{figure}
    \centering
    \includegraphics[width=50mm]{../../resources/img/logo.png}
\end{figure}

\title{College of Engineering, Construction and Living Sciences\\Bachelor of Information Technology\\ID511001: Programming 2\\Level 5, Credits 15\\\textbf{Project 2}}
\date{}
\maketitle

\section*{Assessment Overview}
In this \textbf{individual} assessment, you will design and develop two \textbf{Windows Forms Applications} using \textbf{C\#}. 

\section*{Learning Outcomes}
At the successful completion of this course, learners will be able to:
\begin{enumerate}
    \item Build interactive, event-driven GUI applications using pre-built components.
    \item Declare and implement user-defined classes using encapsulation, inheritance and polymorphism.
\end{enumerate}

\section*{Assessments}
\renewcommand{\arraystretch}{1.5}
\begin{tabular}{|c|c|c|c|}
	\hline
	\textbf{Assessment}                                 & \textbf{Weighting} & \textbf{Due Date}            & \textbf{Learning Outcomes} \\ \hline
	\small Project 1 & \small 25\%        & \small \small 13-11-2024 (Wednesday at 4.59 PM)  & \small 1, 2                   \\ \hline
	\small Project 2  & \small 35\%        & \small 18-09-2024 (Wednesday at 4.59 PM)   & \small 1, 2                   \\ \hline
	\small Theory Examination                        & \small 30\%        & \small 18-11-2024 (Monday at 12.00 PM)  & \small 1, 2                   \\ \hline
	\small Classroom Tasks                      & \small 10\%        & \small 18-09-2024 (Wednesday at 4.59 PM)  & \small 1, 2                   \\ \hline
\end{tabular} 

\section*{Conditions of Assessment}
You will complete this assessment during your learner-managed time. However, there will be time during class to discuss the requirements and your progress on this assessment. This assessment will need to be completed by \textbf{Wednesday, 18 September 2024} at \textbf{4.59 PM}.

\section*{Pass Criteria}
This assessment is criterion-referenced (CRA) with a cumulative pass mark of \textbf{50\%} over all assessments in \textbf{ID511001: Programming 2}.

\section*{Authenticity}
All parts of your submitted assessment \textbf{must} be completely your work. Do your best to complete this assessment without using an \textbf{AI generative tool}. You need to demonstrate to the course lecturer that you can meet the learning outcome(s) for this assessment. \\
 
 However, if you get stuck, you can use an \textbf{AI generative tool} to help you get unstuck, permitting you to acknowledge that you have used it. In the assessment's repository \textbf{README.md} file, please include what prompt(s) you provided to the \textbf{AI generative tool} and how you used the response(s) to help you with your work. It also applies to code snippets retrieved from \textbf{StackOverflow} and \textbf{GitHub}. \\
 
 Failure to do this may result in a mark of \textbf{zero} for this assessment.

\section*{Policy on Submissions, Extensions, Resubmissions and Resits}
The school's process concerning submissions, extensions, resubmissions and resits complies with \textbf{Otago Polytechnic | Te Pūkenga} policies. Learners can view policies on the \textbf{Otago Polytechnic | Te Pūkenga} website located at \href{https://www.op.ac.nz/about-us/governance-and-management/policies}{https://www.op.ac.nz/about-us/governance-and-management/policies}.

\section*{Submission}
You \textbf{must} submit all application files via \textbf{GitHub Classroom}. Here is the URL to the repository you will use for your submission – \href{https://classroom.github.com/a/o7D2CGoa}{https://classroom.github.com/a/o7D2CGoa}. If you do not have not one, create a \textbf{.gitignore} and add the ignored files in this resource - \href{https://raw.githubusercontent.com/github/gitignore/main/VisualStudio.gitignore}{https://raw.githubusercontent.com/github/gitignore/main/VisualStudio.gitignore}. Create a branch called \textbf{project-2}. The latest application files in the \textbf{project-2} branch will be used to mark against the \textbf{Functionality} criterion. Please test before you submit. Partial marks \textbf{will not} be given for incomplete functionality. Late submissions will incur a \textbf{10\% penalty per day}, rolling over at \textbf{5:00 PM}.

\section*{Extensions}
Familiarise yourself with the assessment due date. Extensions will \textbf{only} be granted if you are unable to complete the assessment by the due date because of \textbf{unforeseen circumstances outside your control}. The length of the extension granted will depend on the circumstances and \textbf{must} be negotiated with the course lecturer before the assessment due date. A medical certificate or support letter may be needed. Extensions will not be granted for poor time management or pressure of other assessments.

\section*{Resits}
Resits and reassessments \textbf{are not} applicable in \textbf{ID511001: Programming 2}.

\section*{Instructions}
You will need to submit applications and documentation that meet the following requirements:\\

\subsection*{Functionality - Learning Outcome 1 (50\%)}

\subsubsection*{Student Management System}

The application needs to open without code or file structure modification in \textbf{Visual Studio}.

\subsubsection*{Milestone One - Due Sunday, 25 August 2024 (Week 5) at 4.59 PM}

\begin{itemize}
    \item \textbf{Institution.cs}. \textbf{public class Institution} has the following \textbf{private} fields: \textbf{name} of type \textbf{string}, \textbf{region} of type \textbf{string} and \textbf{country} of type \textbf{string}. This class has a \textbf{public constructor} that takes in the following parameters: \textbf{name} of type \textbf{string}, \textbf{region} of type \textbf{string} and \textbf{country} of type \textbf{string}. 
    \item \textbf{Department.cs}. \textbf{public class Department} has the following \textbf{private} fields: \textbf{institution} of type \textbf{Institution} and \textbf{name} of type \textbf{string}. This class has a \textbf{public constructor} that takes in the following parameters: \textbf{institution} of type \textbf{Institution} and \textbf{name} of type \textbf{string}.
    \item \textbf{Course.cs}. \textbf{public class Course} has the following \textbf{private} fields: \textbf{department} of type \textbf{Department}, \textbf{code} of type \textbf{string}, \textbf{name} of type \textbf{string}, \textbf{description} of type \textbf{string}, \textbf{credits} of type \textbf{int} and \textbf{fees} of type \textbf{double}. This class has a \textbf{public constructor} that takes in the following parameters: \textbf{department} of type \textbf{Department}, \textbf{code} of type \textbf{string}, \textbf{name} of type \textbf{string}, \textbf{description} of type \textbf{string}, \textbf{credits} of type \textbf{int} and \textbf{fees} of type \textbf{double}.
    \item \textbf{Utils.cs}. \textbf{public static class Utils} has the following \textbf{public static} methods:
    \begin{itemize}
        \item \textbf{SeedInstitutions()} with the return type of \textbf{List$<$Institution$>$}. This method seeds a \textbf{List$<$Institution$>$} with \textbf{three} \textbf{Institution} objects.
        \item \textbf{SeedDepartments()} with the return type of \textbf{List$<$Department$>$}. This method seeds a \textbf{List$<$Department$>$} with \textbf{three} \textbf{Department} objects.
        \item \textbf{SeedCourses()} with the return type of \textbf{List$<$Course$>$}. This method seeds a \textbf{List$<$Course$>$} with \textbf{three} \textbf{Course} objects.
    \end{itemize}
\end{itemize}

\subsubsection*{Milestone Two - Due Sunday, 01 September 2024 (Week 6) at 4.59 PM}

\begin{itemize}
    \item \textbf{CourseAssessmentMark.cs}. \textbf{public class CourseAssessmentMark} has the following \textbf{private} fields: \textbf{course} of type \textbf{Course} and \textbf{assessmentMarks} of type \textbf{List$<$int$>$}. Also, this class has the following \textbf{public} methods:
    \begin{itemize}
        \item \textbf{CourseAssessmentMark()} with no return type and two parameters: \textbf{course} of type \textbf{Course} and \textbf{assessmentMarks} of type \textbf{List$<$int$>$}. 
        \item \textbf{GetAllMarks()} with the return type of \textbf{List$<$int$>$}. This method returns all assessment marks.
        \item \textbf{GetAllGrades()} with the return type of \textbf{List$<$string$>$}. This method returns all assessment grades.
        \item \textbf{GetHighestMarks()} with the return type of \textbf{List$<$int$>$}. This method returns the highest passing assessment mark(s).
        \item \textbf{GetLowestMarks()} with the return type of \textbf{List$<$int$>$}. This method returns the lowest passing assessment mark(s). 
        \item \textbf{GetFailMarks()} with the return type of \textbf{List$<$int$>$}. This method returns the fail assessment mark(s).
        \item \textbf{GetAverageMark()} with the return type of \textbf{double}. This method returns the average assessment mark rounded to two decimal places.
        \item \textbf{GetAverageGrade()} with the return type of \textbf{string}. This method returns the average assessment grade.
    \end{itemize}
    For more information on how to calculate the highest, lowest and fail marks, refer to the \textbf{grade table} in the \textbf{Additional Information} section below.
\end{itemize}

\subsubsection*{Milestone Three - Due wednesday, 4 September 2024 (Week 7) at 4.59 PM}

\begin{itemize}
    \item The application needs to contain the following \textbf{enums}:
    \begin{verbatim}
public enum EPosition
{
    LECTURER = 0,
    SENIOR_LECTURER = 1,
    PRINCIPAL_LECTURER = 2,
    ASSOCIATE_PROFESSOR = 3,
    PROFESSOR = 4
}
            
public enum ESalary
{
    LECTURER_SALARY = 85000,
    SENIOR_LECTURER_SALARY = 100000,
    PRINCIPAL_LECTURER_SALARY = 115000,
    ASSOCIATE_PROFESSOR_SALARY = 130000,
    PROFESSOR_SALARY = 145000
}
    \end{verbatim}
    \item \textbf{Person.cs}. \textbf{public class Person} has the following \textbf{protected} fields: \textbf{id} of type \textbf{int}, \textbf{firstName} of type \textbf{string} and \textbf{lastName} of type \textbf{string}. This class has a \textbf{public constructor} that takes in the following parameters: \textbf{id} of type \textbf{int}, \textbf{firstName} of type \textbf{string} and \textbf{lastName} of type \textbf{string}.
    \item \textbf{Learner.cs}. \textbf{public class Learner} inherits from \textbf{Person} and has one additional \textbf{private} field: \textbf{courseAssessmentMarks} of type \textbf{CourseAssessmentMark}. This class has a \textbf{public constructor} that takes in the following parameters: \textbf{id} of type \textbf{int}, \textbf{firstName} of type \textbf{string}, \textbf{lastName} of type \textbf{string} and \textbf{courseAssessmentMarks} of type \textbf{CourseAssessmentMark}. 
    \item \textbf{Lecturer.cs}. \textbf{public class Lecturer} inherits from \textbf{Person} and has three additional \textbf{private} fields: \textbf{position} of type \textbf{EPosition}, \textbf{salary} of type \textbf{ESalary} and \textbf{course} of type \textbf{Course}. This class has a \textbf{public constructor} that takes in the following parameters: \textbf{id} of type \textbf{int}, \textbf{firstName} of type \textbf{string}, \textbf{lastName} of type \textbf{string}, \textbf{position} of type \textbf{EPosition}, \textbf{salary} of type \textbf{ESalary} and \textbf{course} of type \textbf{Course}. 
\end{itemize}

\subsubsection*{Milestone Four - Due Sunday, 8 September 2024 (Week 8) at 4.59 PM}

\begin{itemize}
    \item \textbf{Utils.cs}. \textbf{public static class Utils} has the following \textbf{public static} methods:
        \begin{itemize}
            \item In the \textbf{assessments $>$ project 2} directory of the \textbf{course materials} repository, you will find the implementation of the following method in the \textbf{read-from-file-learners.txt} file.
            \textbf{ReadFromFile()} with no return type and three parameters: \textbf{filePath} of type \textbf{string}, \textbf{learners} of type \textbf{List$<$Learner$>$} and \textbf{isAttendance} of type \textbf{bool}. This method reads the \textbf{learners.txt} file and populates the \textbf{learners} parameter. The \textbf{learners.txt} file contains the following information:\\
            \begin{verbatim}
1,John,Doe,0,100,100,95,10,0
2,Jane,Doe,0,45,35,45,75,65
3,Grayson,Orr,1,50,60,75,80,55
4,Joe,Blogs,1,10,20,30,70,80
5,Bob,Brown,2,75,82,95,55,10
            \end{verbatim}
            What do the numbers represent? The first number is the \textbf{id}, the second number is the \textbf{course} and the remaining numbers are the \textbf{assessment marks 1-5}. \textbf{Note:} You will use the second number to access a \textbf{Course} object from the \textbf{courses} list.
        \item \textbf{ReadFromFile()} with no return type and two parameters: \textbf{filePath} of type \textbf{string} and \textbf{lecturers} of type \textbf{List$<$Lecturer$>$}. This method reads the \textbf{lecturers.txt} file and populates the \textbf{lecturers} parameter. The \textbf{lecturers.txt} file contains the following information:\\
        \begin{verbatim}
1,Graydon,Ore,1,100000,0
2,Aidan,Moscow,2,115000,1
3,Jon,Seena,0,85000,2
        \end{verbatim}
        What do the numbers represent? The first number is the \textbf{id}, the second number is the \textbf{position}, the third number is the \textbf{salary} and the fourth number is the \textbf{course}. \textbf{Note:} You will use the second number to access the enum \textbf{EPosition}, the third number to access the enum \textbf{ESalary} and fourth number to access a \textbf{Course} object from the \textbf{courses} list.
    \end{itemize}
\end{itemize}

\subsubsection*{Milestone Five - Due Wednesday, 11 September 2024 (Week 9) at 4.59 PM}

\begin{itemize}
    \item \textbf{Form1.cs}. \textbf{public class Form1} manages the user interface. This class needs to account for the following functionality:
    \begin{itemize}
        \item Seeding the \textbf{institutions}, \textbf{departments} and \textbf{courses} lists.
        \item Reading the \textbf{learners.txt} and \textbf{lecturers.txt} files. When calling the \textbf{ReadFromFile()} method, the \textbf{isAttendance} parameter needs to be \textbf{false}.
        \item Displaying course details. This needs to display the following details in a \textbf{DataGridView}:
        \begin{itemize}
            \item course \textbf{code} and \textbf{name} in this format - \textbf{code: name}
            \item course \textbf{description}
            \item course \textbf{credits}
            \item course \textbf{fees}
            \item institution \textbf{name}, \textbf{region} and \textbf{country} in this format - \textbf{name, region, country}
            \item department \textbf{name}
        \end{itemize}
        \item Displaying all marks. This needs to display the following details in a \textbf{DataGridView}:
        \begin{itemize}
            \item learner \textbf{id}
            \item learner \textbf{first name} and \textbf{last name} in this format - \textbf{first name last name}
            \item course \textbf{code} and \textbf{name} in this format - \textbf{code: name}
            \item learner \textbf{assessment marks 1-5} in this format - \textbf{mark 1, mark 2, mark 3, mark 4, mark 5}
        \end{itemize}
        \item Displaying all grades. This needs to display the following details in a \textbf{DataGridView}:
        \begin{itemize}
            \item learner \textbf{id}
            \item learner \textbf{first name} and \textbf{last name} in this format - \textbf{first name last name}
            \item course \textbf{code} and \textbf{name} in this format - \textbf{code: name}
            \item learner  \textbf{assessment grades 1-5} in this format - \textbf{grade 1, grade 2, grade 3, grade 4, grade 5}
        \end{itemize}
        \item Displaying highest marks. This needs to display the following details in a \textbf{DataGridView}:
        \begin{itemize}
            \item learner \textbf{id}
            \item learner \textbf{first name} and \textbf{last name} in this format - \textbf{first name last name}
            \item course \textbf{code} and \textbf{name} in this format - \textbf{code: name}
            \item learner \textbf{highest assessment mark(s)}
        \end{itemize}
        \item Displaying lowest marks. This needs to display the following details in a \textbf{DataGridView}:
        \begin{itemize}
            \item learner \textbf{id}
            \item learner \textbf{first name} and \textbf{last name} in this format - \textbf{first name last name}
            \item course \textbf{code} and \textbf{name} in this format - \textbf{code: name}
            \item learner \textbf{lowest assessment mark(s)}
        \end{itemize}
        \item Displaying fail marks. This needs to display the following details in a \textbf{DataGridView}:
        \begin{itemize}
            \item learner \textbf{id}
            \item learner \textbf{first name} and \textbf{last name} in this format - \textbf{first name last name}
            \item course \textbf{code} and \textbf{name} in this format - \textbf{code: name}
            \item learner \textbf{fail assessment mark(s)}
        \end{itemize}
        \item Displaying average marks. This needs to display the following details in a \textbf{DataGridView}:
        \begin{itemize}
            \item learner \textbf{id}
            \item learner \textbf{first name} and \textbf{last name} in this format - \textbf{first name last name}
            \item course \textbf{code} and \textbf{name} in this format - \textbf{code: name}
            \item learner \textbf{average assessment mark}
        \end{itemize}
        \item Displaying average grades. This needs to display the following details in a \textbf{DataGridView}:
        \begin{itemize}
            \item learner \textbf{id}
            \item learner \textbf{first name} and \textbf{last name} in this format - \textbf{first name last name}
            \item course \textbf{code} and \textbf{name} in this format - \textbf{code: name}
            \item learner \textbf{average assessment grade}
        \end{itemize}
        \item Displaying lecturer details. This needs to display the following details in a \textbf{DataGridView}:
        \begin{itemize}
            \item lecturer \textbf{id}
            \item lecturer \textbf{first name} and \textbf{last name} in this format - \textbf{first name last name}
            \item lecturer \textbf{position}
            \item institution \textbf{name}, \textbf{region} and \textbf{country} in this format - \textbf{name, region, country}
            \item department \textbf{name}
            \item course \textbf{code} and \textbf{name} in this format - \textbf{code: name}
            \item lecturer \textbf{salary}
        \end{itemize}
        \item Adding a learner. When adding a learner, the \textbf{id} is auto-generated and unique. Prompt the user to enter the following details: \textbf{first name}, \textbf{last name}, \textbf{course} and \textbf{assessment marks 1-5}. Implement the following validation:
        \begin{itemize}
            \item \textbf{first name} and \textbf{last name} are not empty or, contain numbers and special characters.
            \item \textbf{course} is a valid number.
            \item \textbf{assessment marks 1-5} are between 0 and 100.
        \end{itemize} 
        If the validation is successful, add the learner to the \textbf{learners} list, write the learner to the \textbf{learners.txt} file and display a success message. Otherwise, display an error message.                          
        \item Adding a lecturer. When adding a lecturer, the \textbf{id} is auto-generated and unique. The \textbf{salary} is calculated based on the \textbf{position}. Prompt the user to enter the following details: \textbf{first name}, \textbf{last name}, \textbf{position} and \textbf{course}. Implement the following validation:
        \begin{itemize}
            \item \textbf{first name} and \textbf{last name} are not empty or, contain numbers and special characters.
            \item \textbf{position} and \textbf{course} are valid numbers.
        \end{itemize}
        If the validation is successful, add the lecturer to the \textbf{lecturers} list, write the lecturer to the \textbf{lecturers.txt} file and display a success message. Otherwise, display an error message.
        \item Removing a lecturer. When removing a lecturer, prompt the user to enter the \textbf{id} of the learner to remove. If the \textbf{id} is valid, prompt the user to confirm the removal. If the user confirms the removal, remove the lecturer from the \textbf{lecturers} list and \textbf{lecturer.txt} file, and display a success message. Otherwise, display an error message.
        \item Exiting the application
    \end{itemize} 
    If the user enters an invalid option, display an error message and prompt the user to enter a valid option.
\end{itemize}


\subsubsection*{Attendance Tracker}

The application needs to open without code or file structure modification in \textbf{Visual Studio}.

\subsubsection*{Milestone Six - Due Wednesday, 18 September 2024 (Week 10) at 4.59 PM}

\begin{itemize}
    \item Add a \textbf{Project Reference} to the \textbf{Student Management System} application.
    \item \textbf{Learner.cs}. In this class, you will add an additional \textbf{private} field: \textbf{attendance} of type \textbf{int} and a \textbf{public constructor} that takes in the following parameters: \textbf{id} of type \textbf{int}, \textbf{firstName} of type \textbf{string}, \textbf{lastName} of type \textbf{string} and \textbf{attendance} of type \textbf{int}. \textbf{Note:} This class should have two \textbf{private} fields and two \textbf{public constructors}.
    \item \textbf{Form1.cs}. \textbf{public class Form1} manages the user interface. This class needs to account for the following functionality:
    \begin{itemize}
        \item Reading the \textbf{attendance.txt} file. \textbf{Note:} Uncomment the two lines of code in this file. When calling the \textbf{ReadFromFile()} method, the \textbf{isAttendance} parameter needs to be \textbf{true}.  
        \item Displaying attendance. This needs to display the following details in a \textbf{DataGridView}:
        \begin{itemize}
            \item learner \textbf{id}
            \item learner \textbf{first name} and \textbf{last name} in this format - \textbf{first name last name}
            \item learner \textbf{percentage}
        \end{itemize}
        \item Display learners at risk. A learner at risk is someone who has an attendance percentage of less than 50\%. This needs to display the following learner \textbf{first name} and \textbf{last name} in this format - \textbf{first name last name} in a \textbf{ListBox}. \textbf{Note:} You cannot deviate from the required format.
    \end{itemize}
\end{itemize}

\subsection*{Code Quality and Best Practices - Learning Outcome 2 (45\%)}
\begin{itemize}
    \item A \textbf{Visual Studio} \textbf{.gitignore} file is used. 
    \item Appropriate naming of files, variables, methods and classes.
    \item Idiomatic use of object-oriented principles, values, control flow, data structures and in-built functions.
    \item Efficient algorithmic approach.
    \item Sufficient modularity.
    \item Each file has an \textbf{XML documentation comment} located at the top of the file. In the \textbf{assessment} directory of the \textbf{course materials} repository, you will find an \textbf{XML documentation comment} example in the \textbf{xml-documentation-comment.txt} file.
    \item Formatted code.
    \item No dead or unused code.
\end{itemize}

\subsection*{Documentation - Learning Outcome 2 (5\%)}
\begin{itemize}
    \item Provide the following in your repository \textbf{README.md} file:
    \begin{itemize}
        \item A class diagram of your applications.
    \end{itemize}
\end{itemize}

\subsection*{Additional Information}
\begin{itemize}
    \item Exemplars are available in \textbf{assessments $>$ project 2} directory of the \textbf{course materials} repository.
    \item You may add additional classes and methods. 
    \item Grade and mark range table:\\\\
    \renewcommand{\arraystretch}{1.5}
    \begin{tabular}{|c|c|}
        \hline
        \textbf{Grade} & \textbf{Mark Range} \\ \hline
        A+ & 90-100  \\ \hline
        A & 85-89  \\ \hline
        A- & 80-84 \\ \hline
        B+ & 75-79   \\ \hline
        B & 70-74  \\ \hline
        B- & 65-69  \\ \hline
        C+ & 60-64  \\ \hline
        C & 55-59 \\ \hline
        C- & 50-54 (passing assessment marks)  \\ \hline
        D & 40-49 (fail assessment marks)   \\ \hline
        E & 0-39 (fail assessment marks)   \\ \hline
    \end{tabular}
    \item \textbf{Do not} rewrite your \textbf{Git} history. It is important that the course lecturer can see how you worked on your assessment over time.
\end{itemize}

\end{document}