% Author: Grayson Orr
% Course: ID511001: Programming 2

\documentclass{article}
\author{}

\usepackage{graphicx}
\usepackage{wrapfig}
\usepackage{enumerate}
\usepackage{hyperref}
\usepackage[margin = 2.25cm]{geometry}
\usepackage[table]{xcolor}
\usepackage{soul}
\usepackage{fancyhdr}
\usepackage{fancyvrb}
\hypersetup{
  colorlinks = true,
  urlcolor = blue
}
\setlength\parindent{0pt}
\pagestyle{fancy}
\fancyhf{}
\rhead{College of Engineering, Construction and Living Sciences\\Bachelor of Information Technology}
\lfoot{Project 2 (C\# Windows Forms App): Pong\\Version 2, Semester One, 2023}
\rfoot{\thepage}
 
\begin{document}

\begin{figure}
    \centering
    \includegraphics[width=50mm]{../../resources/img/logo.png}
\end{figure}

\title{College of Engineering, Construction and Living Sciences\\Bachelor of Information Technology\\ID511001: Programming 2\\Level 5, Credits 15\\\textbf{Project 2 (C\# Windows Forms App): Pong}}
\date{}
\maketitle

\section*{Assessment Overview}
In this assessment, you will design and develop a pong \textbf{Windows Forms App} using \textbf{C\#}.

\section*{Learning Outcomes}
At the successful completion of this course, learners will be able to:
\begin{enumerate}
    \item Build interactive, event-driven GUI applications using pre-built components.
    \item Declare and implement user-defined classes using encapsulation, inheritance and polymorphism.
\end{enumerate} 

\section*{Assessments}
\renewcommand{\arraystretch}{1.5}
\begin{tabular}{|c|c|c|c|}
	\hline
	\textbf{Assessment}                                 & \textbf{Weighting} & \textbf{Due Date}            & \textbf{Learning Outcomes} \\ \hline
	\small Project 1 (C\# Console App): Learner Gradebook  & \small 25\%        & \small 26-04-2023 (Wednesday at 4.59 PM)   & \small 1 and 2                   \\ \hline
	\small Project 2 (C\# Windows Forms App): Pong & \small 35\%        & \small 14-06-2023 (Wednesday at 4.59 PM)   & \small 1 and 2                   \\ \hline
	\small Theory Examination                        & \small 30\%        & \small 21-06-2023 (Wednesday at 4.45 PM)  & \small 1 and 2                   \\ \hline
	\small Classroom Tasks                       & \small 10\%        & \small 07-06-2023 (Wednesday at 4.59 PM)  & \small 1 and 2                   \\ \hline
\end{tabular} 

\section*{Conditions of Assessment}
You will complete this assessment during your learner-managed time. However, there will be time during class to discuss the requirements and your progress on this assessment. This assessment will need to be completed by \textbf{Wednesday, 14 June 2023} at \textbf{4.59 PM}.

\section*{Pass Criteria}
This assessment is criterion-referenced (CRA) with a cumulative pass mark of \textbf{50\%} over all assessments in \textbf{ID511001: Programming 2}.

\section*{Authenticity}
All parts of your submitted assessment \textbf{must} be completely your work. If you use code snippets from \textbf{GitHub}, \textbf{StackOverflow} or other online resources, you \textbf{must} reference it appropriately using \textbf{APA 7th edition}. Provide your references in the \textbf{README.md} file in your repository. Failure to do this will result in a mark of \textbf{zero} for this assessment.

\section*{Policy on Submissions, Extensions, Resubmissions and Resits}
The school's process concerning submissions, extensions, resubmissions and resits complies with \textbf{Otago Polytechnic | Te Pūkenga} policies. Learners can view policies on the \textbf{Otago Polytechnic | Te Pūkenga} website located at \href{https://www.op.ac.nz/about-us/governance-and-management/policies}{https://www.op.ac.nz/about-us/governance-and-management/policies}.

\section*{Submission}
You \textbf{must} submit all app files via \textbf{GitHub Classroom}. Here is the URL to the repository you will use for your submission – \href{https://classroom.github.com/a/eFe1Oh97}{https://classroom.github.com/a/eFe1Oh97}.  Create a \textbf{.gitignore} and add the ignored files in this resource - \href{https://raw.githubusercontent.com/github/gitignore/main/VisualStudio.gitignore}{https://raw.githubusercontent.com/github/gitignore/main/VisualStudio.gitignore}. The latest app files in the \textbf{master} or \textbf{main} branch will be used to mark against the \textbf{Functionality} criterion. Please test before you submit. Partial marks \textbf{will not} be given for incomplete functionality. Late submissions will incur a \textbf{10\% penalty per day}, rolling over at \textbf{5:00 PM}.

\section*{Extensions}
Familiarise yourself with the assessment due date. Contact the course lecturer before the due date if you need an extension. If you require more than a week's extension, you will need to provide a medical certificate or support letter from your manager.

\section*{Resubmissions}
Learners may be requested to resubmit an assessment following a rework of part/s of the original assessment. Resubmissions are to be completed within a negotiable short time frame and usually \textbf{must} be completed within the timing of the course to which the assessment relates. Resubmissions will be available to learners who have made a genuine attempt at the first assessment opportunity and achieved a \textbf{D grade (40-49\%)}. The maximum grade awarded for resubmission will be \textbf{C-}.

\section*{Resits}
Resits and reassessments \textbf{are not} applicable in \textbf{ID511001: Programming 2}.

\section*{Instructions}
You will need to submit an app and documentation that meet the following requirements:\\

\subsection*{Functionality - Learning Outcomes 1 and 2 (50\%)}
\begin{itemize}
    \item The app \textbf{must} open without code or file structure modification in \textbf{Visual Studio}.
    \item The game \textbf{must} be driven by one \textbf{Timer} and begins when the user presses the \textbf{space bar} key.
    \item The ball and two paddles \textbf{must} be created using the \textbf{Graphics} class. 
    \item The ball \textbf{must} bounce/collide off the top and bottom of the screen, and paddles.
    \item The paddles \textbf{must} move vertically but not exceed the top and bottom of the screen.
    \item The user controls the left paddle via the \textbf{up} and \textbf{down} keys. The computer controls the right paddle. It is acceptable for the right paddle to follow the ball's position. However, other solutions are encouraged.
    \item Double buffering to prevent the ball and paddles from flickering.
    \item A scoring system. When the ball collides with the left and right-hand side of the screen, one point is given to either the user or computer. The game is over when either score is 10. 
    \item Display the user and computer's score using the \textbf{DrawString} method.
    \item A highscore system. When the game is over, appropriate feedback \textbf{must} be displayed to the user, i.e., \textbf{"You win!"} or \textbf{"You lose!"}, the user and computer's scores are saved, i.e., written to a text file. Read the scores from the text file and display the last five to the user.
    \item Play a sound when:
    \begin{itemize}
        \item The ball bounces off the paddle, and top and bottom of the screen.
        \item The user wins.
        \item The user loses.
    \end{itemize}
    \textbf{Note:} These sounds \textbf{must} be unique.
    \item An ability to restart and pause a game.
    \item Randomise the colour of the ball and paddles.
\end{itemize}

\subsection*{Code Elegance - Learning Outcomes 1 and 2  (40\%)}
\begin{itemize}
    \item Adhere to the four principles of \textbf{OO}, i.e., encapsulation, abstraction, inheritance and polymorphism.
    \item Use of intermediate variables, constants and enumerations.
    \item Idiomatic use of control flow, data structures and in-built functions.
    \item Efficient algorithmic approach.
    \item Sufficient modularity.
    \item Each method and class \textbf{must} have a header comment located immediately before its declaration.
    \item In-line comments where required. 
    \item Project files, i.e., \textbf{.cs} files are formatted. 
    \item No dead or unused code.
\end{itemize}

\subsection*{Documentation and Git Usage - Learning Outcomes 1 and 2 (10\%)}
\begin{itemize}
    \item Provide the following in your repository \textbf{README.md} file:
    \begin{itemize}
        \item The app's class diagram created in \textbf{Visual Studio}.
        \item Known bugs if applicable.
    \end{itemize}
    \item Commit at least \textbf{20} times per week.
    \item Commit messages \textbf{must} be formatted using the recommended conventions and reflect the context of each functional requirement change.
\end{itemize}

\subsection*{Additional Information}
\begin{itemize}
    \item \textbf{Do not} rewrite your \textbf{Git} history. It is important that the course lecturer can see how you worked on your assessment over time.
    \item When the user presses a key, i.e., up or down, a \textbf{KeyDown} event is generated. For the \textbf{Form1's} \textbf{KeyDown} event, the method signature is:
    \begin{Verbatim}[tabsize=2]
		private void Form1_KeyDown(object sender, KeyEventArgs e) {}
	\end{Verbatim}
	The argument you will be interested in is \textbf{KeyEventArgs e} which is the value of the pressed key. The arrow key values are \textbf{Keys.Left}, \textbf{Keys.Right}, \textbf{Keys.Up} and \textbf{Keys.Down}. In the \textbf{Form1\_KeyDown} method, you can use a \textbf{switch} statement. For example:
	\begin{Verbatim}[tabsize=2]
		switch (e.KeyCode) 
		{
			case Keys.Left:
				// Do something
				break;
			case Keys.Right:
				// Do something
				break;
			case Keys.Up:
				// Do something
				break;
			case Keys.Down:
				// Do something
				break;
			default:
				// Do something
				break;
		}
	\end{Verbatim}
	\textbf{Note:} The \textbf{Form1's} \textbf{KeyPreview} event \textbf{must} be set to \textbf{True}. Otherwise, \textbf{Form1} will not respond to the \textbf{KeyDown} event.
\end{itemize}

\end{document}