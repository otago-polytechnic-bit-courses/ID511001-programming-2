% Author: Grayson Orr
% Course: ID511001: Programming 2

\documentclass{article}
\author{}

\usepackage{graphicx}
\usepackage{wrapfig}
\usepackage{enumerate}
\usepackage{hyperref}
\usepackage[margin = 2.25cm]{geometry}
\usepackage[table]{xcolor}
\usepackage{soul}
\usepackage{fancyhdr}
\usepackage{fancyvrb}
\hypersetup{
  colorlinks = true,
  urlcolor = blue
}
\setlength\parindent{0pt}
\pagestyle{fancy}
\fancyhf{}
\rhead{College of Engineering, Construction \& Living Sciences\\Bachelor of Information Technology}
\lfoot{Project 2: Space Invaders\\Version 1, Semester Two, 2022}
\rfoot{\thepage}
 
\begin{document}

\begin{figure}
    \centering
    \includegraphics[width=50mm]{../../resources/img/logo.png}
\end{figure}

\title{College of Engineering, Construction \& Living Sciences\\Bachelor of Information Technology\\ID511001: Programming 2\\Level 5, Credits 15\\\textbf{Project 2: Space Invaders}}
\date{}
\maketitle

\section*{Assessment Overview}
In this assessment, you will design \& develop a GUI implementation of the classic arcade game \textbf{Space Invaders}.

\section*{Learning Outcome}
At the successful completion of this course, learners will be able to:
\begin{enumerate}
    \item Build interactive, event-driven GUI applications using pre-built components.
    \item Declare \& implement user-defined classes using encapsulation, inheritance \& polymorphism.
\end{enumerate}

\section*{Assessments}
\renewcommand{\arraystretch}{1.5}
\begin{tabular}{|c|c|c|c|}
    \hline
    \textbf{Assessment}                                 & \textbf{Weighting} & \textbf{Due Date}            & \textbf{Learning Outcomes} \\ \hline
    \small Project 1: Pong & \small 25\%        & \small 04-11-2022 (Friday at 4.59 PM)   & \small 1 \& 2                   \\ \hline
    \small Project 2: Space Invaders & \small 35\%        & \small 04-11-2022 (Friday at 4.59 PM)   & \small 1 \& 2                   \\ \hline
    \small Theory Examination                        & \small 30\%        & \small 18-11-2022 (Friday at 4.59 PM)  & \small 1 \& 2                   \\ \hline
    \small Classroom Tasks                       & \small 10\%        & \small 21-10-2022 (Friday at 4.59 PM)  & \small 1 \& 2                   \\ \hline
\end{tabular}

\section*{Conditions of Assessment}
You will complete this assessment during your learner-managed time. However, there will be time during class to discuss the requirements \& your progress on this assessment. This assessment will need to be completed by \textbf{Friday, 04 November 2022} at \textbf{4.59 PM}.

\section*{Pass Criteria}
This assessment is criterion-referenced (CRA) with a cumulative pass mark of \textbf{50\%} over all assessments in \textbf{ID511001: Programming 2}.

\section*{Authenticity}
All parts of your submitted assessment \textbf{must} be completely your work. If you use code snippets from \textbf{GitHub}, \textbf{StackOverflow} or other online resources, you \textbf{must} reference it appropriately using \textbf{APA 7th edition}. Provide your references in the \textbf{README.md} file in your repository. Failure to do this will result in a mark of \textbf{zero} for this assessment.

\section*{Policy on Submissions, Extensions, Resubmissions \& Resits}
The school's process concerning submissions, extensions, resubmissions \& resits complies with \textbf{Otago Polytechnic} policies. Learners can view policies on the \textbf{Otago Polytechnic} website located at \href{https://www.op.ac.nz/about-us/governance-and-management/policies}{https://www.op.ac.nz/about-us/governance-and-management/policies}.

\section*{Submission}
You \textbf{must} submit all project files via \textbf{GitHub Classroom}. Here is the URL to the repository you will use for your submission – \href{https://classroom.github.com/a/J5gYpMTC}{https://classroom.github.com/a/J5gYpMTC}.  Create a \textbf{.gitignore} \& add the ignored files in this resource - \href{https://raw.githubusercontent.com/github/gitignore/main/VisualStudio.gitignore}{https://raw.githubusercontent.com/github/gitignore/main/VisualStudio.gitignore}. The latest project files in the \textbf{master} or \textbf{main} branch will be used to mark against the \textbf{Functionality} criterion. Please test your \textbf{master} or \textbf{main} branch application before you submit. Partial marks \textbf{will not} be given for incomplete functionality. Late submissions will incur a \textbf{10\% penalty per day}, rolling over at \textbf{5:00 PM}.

\section*{Extensions}
Familiarise yourself with the assessment due date. Contact the course lecturer before the due date if you need an extension. If you require more than a week's extension, you will need to provide a medical certificate or support letter from your manager.

\section*{Resubmissions}
Learners may be requested to resubmit an assessment following a rework of part/s of the original assessment. Resubmissions are to be completed within a negotiable short time frame \& usually \textbf{must} be completed within the timing of the course to which the assessment relates. Resubmissions will be available to learners who have made a genuine attempt at the first assessment opportunity \& achieved a \textbf{D grade (40-49\%)}. The maximum grade awarded for resubmission will be \textbf{C-}.

\section*{Resits}
Resits \& reassessments \textbf{are not} applicable in \textbf{ID511001: Programming 2}.

\section*{Instructions}
You will need to submit a project \& documentation that meet the following requirements:\\

\textbf{Note:} Independent research requirements are \hl{highlighted yellow}.

\subsection*{Functionality - Learning Outcomes 1 \& 2 (40\%)}
\begin{itemize}
    \item Project \textbf{must} open with code or file structure modification in \textbf{Visual Studio}.
    \item A game of \textbf{Space Invaders} \textbf{must} be driven by one \textbf{Timer}.
    \item Mother ship
    \begin{itemize}
        \item Move left \& right in response to a key press.
        \item Fire in response to a key press.
        \item Have no more than 15 missiles in the air at any one time.
        \item Be destroyed if hit by an enemy ship's bomb.
    \end{itemize}
    \item Mother ship missile
    \begin{itemize}
        \item Be fired from the top of the mother ship.
        \item Move only upward \& in a straight line.
        \item Live for a random number of timer ticks between 1 \& 70.
        \item Be destroyed if they hit an enemy ship.
    \end{itemize}
    \item Enemy ship
    \begin{itemize}
        \item Start in a grid of four rows. Each row has ten enemy ships.
        \item Move left to right in lock step (i.e. the entire grid moves) at an appropriate speed. Reverse the enemy ships' direction when the leftmost or rightmost edge of the grid reaches the edge of the screen.
        \item Drop bombs at an appropriate frequency (see details below).
        \item An individual enemy ship may drop a bomb only if there are no other enemy ships in front of them in its column.
        \item Be destroyed when hit by a missile from the mother ship.
    \end{itemize}
    \item Enemy ship bombs
    \begin{itemize}
        \item Be fired by an enemy ship from the closest row to the mother ship. 
        \item Move only downward \& in a straight line.
        \item Live for a random number of timer ticks between 1 \& 70.
        \item Be fired probabilistically. That is, at each time interval, each enemy ship who can fire will have a 1/100 chance of doing so. The logic for this is:
        \begin{Verbatim}[tabsize=2]
        For each enemy ship who can drop a bomb
            Select a random number between 0 & 99
                If that number is 99 then drop a bomb   
        \end{Verbatim}
        \textbf{Note:} You do not have to use 99 here. Using any single value will achieve the same result
    \end{itemize}
    \item A scoring system. The user wins when the mother ship destroys all enemy ships. The user loses when the mother ships is hit by an enemy ship's bomb, or if an enemy ship collides with the bottom of the screen. Appropriate feedback \textbf{must} be displayed to the user, i.e., \textbf{"You win!"} or \textbf{"You lose!"}.
    \item \hl{\textbf{Independent Research:}}
    \begin{itemize}
        \item A highscore system. When the game is over, the user's score is saved, i.e., written to a text file. Read the user's scores from the text file and display the last five to the user.
        \item Play a sound when:
        \begin{itemize}
            \item The mother ship fires a missile.
            \item A missile destroys an enemy ship.
            \item An enemy ship detroys the mother ship.
        \end{itemize}
        \textbf{Note:} These sounds \textbf{must} be unique.
        \item An ability to play a new game, restart a current game \& pause a current game.
    \end{itemize}
\end{itemize}

\subsection*{Code Elegance - Learning Outcomes 1 \& 2  (45\%)}
\begin{itemize}
    \item Adhere to the four principles of \textbf{OO}, i.e., encapsulation, abstraction, inheritance \& polymorphism.
    \item Use of intermediate variables, constants \& enumerations.
    \item Idiomatic use of control flow, data structures \& in-built functions.
    \item Efficient algorithmic approach.
    \item Sufficient modularity.
    \item Each method \& class \textbf{must} have a header comment located immediately before its declaration.
    \item In-line comments where required. 
    \item Project files, i.e., \textbf{.cs} files are fomatted. 
    \item No dead or unused code.
\end{itemize}

\subsection*{Documentation \& Git Usage - Learning Outcomes 1 \& 2 (15\%)}
\begin{itemize}
    \item Provide the following in your repository \textbf{README.md} file:
    \begin{itemize}
        \item Your project's UML diagram.
        \item References to used code snippets from \textbf{GitHub}, \textbf{StackOverflow} or other online resources.
        \item Known bugs if applicable.
    \end{itemize}
    \item Commit at least \textbf{20} times per week.
    \item Commit messages \textbf{must} reflect the context of each functional requirement change \& formatted using the naming conventions dicussed in \textbf{Week 1}.
\end{itemize}

\subsection*{Additional Information}
\begin{itemize}
    \item \textbf{Do not} rewrite your \textbf{Git} history. It is important that the course lecturer can see how you worked on your assessment over time.
    \item You \textbf{must} declare an appropriate class structure for the game objects, i.e., mother ship, missiles, enemy ship \& bombs. All game objects will need to know where they are on the screen, what image they are displaying, whether or not they are still alive, how to draw, how to move \& how to tell if something has collided with them. They will also need to be able to create instances of themselves \& free up their memory at the end of the game.
    \item Individual game objects will also need to know other specialised things. For example, an enemy ship \textbf{must} know if they are in the front of the formation to drop a bomb. Missiles \& bombs need to know their lifespan \& how many ticks they have been alive.
    \item To create the enemy ship formation, do not simply use a global list or array of enemy ships. It would be a poor OO design. Instead, create a fleet object so you can have a fleet of enemy ships, a fleet of missiles \& a fleet of bombs. A fleet object can contain a field that is either a list or a one or two dimensional array.
    \item Your game should be controlled by a Controller class, which updates the state of each game object by looping through the appropriate arrays.
    \item You should drive the entire game off of a single-timer. At each tick, call the Controller to run another iteration of the game.
    \item It is important to keep careful track of whether each object is alive or not. All enemy ships start the game alive. When a missile hits them, they are no longer alive \& never regenerated. As discussed below, missiles \& bombs are created \& destroyed repeatedly throughout the game. To display the correct state of the game, use your timer event to move \& draw only those entities who are alive.
    \item To launch a missile from the mother ship, you will need to ensure that the mother ship does not currently have its limit of 15 missiles in the air. When you wish to launch a missile, you can search the fleet object for a missile that is not currently alive. You can mark this missile as alive \& set its coordinates to where the mother ship is. When a missile hits a target, or reaches the end of its lifespan, mark it as not alive. In the Timer, you draw all missiles which are alive. 
    \item You may use a similar approach to that described above to manage the bombs. However, technically, there is no limit to the number of bombs that may be in the air at any time. Practically, the probabilistic firing behaviour of the enemy ships will keep the number of bombs fairly low.
    \item You \textbf{must} determine when a missile hits an enemy ship \& when a bomb hits the mother ship. It is called collision detection, \& there are many interesting algorithms for implementing it. There is a method that belongs to the Rectangle class called IntersectsWith().
\end{itemize}

\end{document}